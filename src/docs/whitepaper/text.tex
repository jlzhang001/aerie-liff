%\RequirePackage{lineno}
\documentclass[12pt]{article}
\usepackage{graphicx,color,epsfig}
\usepackage{fancyvrb}
\usepackage{color}
\usepackage{framed}
\usepackage[ascii]{inputenc}
\usepackage{float}



\makeatletter
\def\PY@reset{\let\PY@it=\relax \let\PY@bf=\relax%
    \let\PY@ul=\relax \let\PY@tc=\relax%
    \let\PY@bc=\relax \let\PY@ff=\relax}
\def\PY@tok#1{\csname PY@tok@#1\endcsname}
\def\PY@toks#1+{\ifx\relax#1\empty\else%
    \PY@tok{#1}\expandafter\PY@toks\fi}
\def\PY@do#1{\PY@bc{\PY@tc{\PY@ul{%
    \PY@it{\PY@bf{\PY@ff{#1}}}}}}}
\def\PY#1#2{\PY@reset\PY@toks#1+\relax+\PY@do{#2}}

\def\PY@tok@gd{\def\PY@tc##1{\textcolor[rgb]{0.63,0.00,0.00}{##1}}}
\def\PY@tok@gu{\let\PY@bf=\textbf\def\PY@tc##1{\textcolor[rgb]{0.50,0.00,0.50}{##1}}}
\def\PY@tok@gt{\def\PY@tc##1{\textcolor[rgb]{0.00,0.25,0.82}{##1}}}
\def\PY@tok@gs{\let\PY@bf=\textbf}
\def\PY@tok@gr{\def\PY@tc##1{\textcolor[rgb]{1.00,0.00,0.00}{##1}}}
\def\PY@tok@cm{\let\PY@it=\textit\def\PY@tc##1{\textcolor[rgb]{0.25,0.50,0.50}{##1}}}
\def\PY@tok@vg{\def\PY@tc##1{\textcolor[rgb]{0.10,0.09,0.49}{##1}}}
\def\PY@tok@m{\def\PY@tc##1{\textcolor[rgb]{0.40,0.40,0.40}{##1}}}
\def\PY@tok@mh{\def\PY@tc##1{\textcolor[rgb]{0.40,0.40,0.40}{##1}}}
\def\PY@tok@go{\def\PY@tc##1{\textcolor[rgb]{0.50,0.50,0.50}{##1}}}
\def\PY@tok@ge{\let\PY@it=\textit}
\def\PY@tok@vc{\def\PY@tc##1{\textcolor[rgb]{0.10,0.09,0.49}{##1}}}
\def\PY@tok@il{\def\PY@tc##1{\textcolor[rgb]{0.40,0.40,0.40}{##1}}}
\def\PY@tok@cs{\let\PY@it=\textit\def\PY@tc##1{\textcolor[rgb]{0.25,0.50,0.50}{##1}}}
\def\PY@tok@cp{\def\PY@tc##1{\textcolor[rgb]{0.74,0.48,0.00}{##1}}}
\def\PY@tok@gi{\def\PY@tc##1{\textcolor[rgb]{0.00,0.63,0.00}{##1}}}
\def\PY@tok@gh{\let\PY@bf=\textbf\def\PY@tc##1{\textcolor[rgb]{0.00,0.00,0.50}{##1}}}
\def\PY@tok@ni{\let\PY@bf=\textbf\def\PY@tc##1{\textcolor[rgb]{0.60,0.60,0.60}{##1}}}
\def\PY@tok@nl{\def\PY@tc##1{\textcolor[rgb]{0.63,0.63,0.00}{##1}}}
\def\PY@tok@nn{\let\PY@bf=\textbf\def\PY@tc##1{\textcolor[rgb]{0.00,0.00,1.00}{##1}}}
\def\PY@tok@no{\def\PY@tc##1{\textcolor[rgb]{0.53,0.00,0.00}{##1}}}
\def\PY@tok@na{\def\PY@tc##1{\textcolor[rgb]{0.49,0.56,0.16}{##1}}}
\def\PY@tok@nb{\def\PY@tc##1{\textcolor[rgb]{0.00,0.50,0.00}{##1}}}
\def\PY@tok@nc{\let\PY@bf=\textbf\def\PY@tc##1{\textcolor[rgb]{0.00,0.00,1.00}{##1}}}
\def\PY@tok@nd{\def\PY@tc##1{\textcolor[rgb]{0.67,0.13,1.00}{##1}}}
\def\PY@tok@ne{\let\PY@bf=\textbf\def\PY@tc##1{\textcolor[rgb]{0.82,0.25,0.23}{##1}}}
\def\PY@tok@nf{\def\PY@tc##1{\textcolor[rgb]{0.00,0.00,1.00}{##1}}}
\def\PY@tok@si{\let\PY@bf=\textbf\def\PY@tc##1{\textcolor[rgb]{0.73,0.40,0.53}{##1}}}
\def\PY@tok@s2{\def\PY@tc##1{\textcolor[rgb]{0.73,0.13,0.13}{##1}}}
\def\PY@tok@vi{\def\PY@tc##1{\textcolor[rgb]{0.10,0.09,0.49}{##1}}}
\def\PY@tok@nt{\let\PY@bf=\textbf\def\PY@tc##1{\textcolor[rgb]{0.00,0.50,0.00}{##1}}}
\def\PY@tok@nv{\def\PY@tc##1{\textcolor[rgb]{0.10,0.09,0.49}{##1}}}
\def\PY@tok@s1{\def\PY@tc##1{\textcolor[rgb]{0.73,0.13,0.13}{##1}}}
\def\PY@tok@sh{\def\PY@tc##1{\textcolor[rgb]{0.73,0.13,0.13}{##1}}}
\def\PY@tok@sc{\def\PY@tc##1{\textcolor[rgb]{0.73,0.13,0.13}{##1}}}
\def\PY@tok@sx{\def\PY@tc##1{\textcolor[rgb]{0.00,0.50,0.00}{##1}}}
\def\PY@tok@bp{\def\PY@tc##1{\textcolor[rgb]{0.00,0.50,0.00}{##1}}}
\def\PY@tok@c1{\let\PY@it=\textit\def\PY@tc##1{\textcolor[rgb]{0.25,0.50,0.50}{##1}}}
\def\PY@tok@kc{\let\PY@bf=\textbf\def\PY@tc##1{\textcolor[rgb]{0.00,0.50,0.00}{##1}}}
\def\PY@tok@c{\let\PY@it=\textit\def\PY@tc##1{\textcolor[rgb]{0.25,0.50,0.50}{##1}}}
\def\PY@tok@mf{\def\PY@tc##1{\textcolor[rgb]{0.40,0.40,0.40}{##1}}}
\def\PY@tok@err{\def\PY@bc##1{\fcolorbox[rgb]{1.00,0.00,0.00}{1,1,1}{##1}}}
\def\PY@tok@kd{\let\PY@bf=\textbf\def\PY@tc##1{\textcolor[rgb]{0.00,0.50,0.00}{##1}}}
\def\PY@tok@ss{\def\PY@tc##1{\textcolor[rgb]{0.10,0.09,0.49}{##1}}}
\def\PY@tok@sr{\def\PY@tc##1{\textcolor[rgb]{0.73,0.40,0.53}{##1}}}
\def\PY@tok@mo{\def\PY@tc##1{\textcolor[rgb]{0.40,0.40,0.40}{##1}}}
\def\PY@tok@kn{\let\PY@bf=\textbf\def\PY@tc##1{\textcolor[rgb]{0.00,0.50,0.00}{##1}}}
\def\PY@tok@mi{\def\PY@tc##1{\textcolor[rgb]{0.40,0.40,0.40}{##1}}}
\def\PY@tok@gp{\let\PY@bf=\textbf\def\PY@tc##1{\textcolor[rgb]{0.00,0.00,0.50}{##1}}}
\def\PY@tok@o{\def\PY@tc##1{\textcolor[rgb]{0.40,0.40,0.40}{##1}}}
\def\PY@tok@kr{\let\PY@bf=\textbf\def\PY@tc##1{\textcolor[rgb]{0.00,0.50,0.00}{##1}}}
\def\PY@tok@s{\def\PY@tc##1{\textcolor[rgb]{0.73,0.13,0.13}{##1}}}
\def\PY@tok@kp{\def\PY@tc##1{\textcolor[rgb]{0.00,0.50,0.00}{##1}}}
\def\PY@tok@w{\def\PY@tc##1{\textcolor[rgb]{0.73,0.73,0.73}{##1}}}
\def\PY@tok@kt{\def\PY@tc##1{\textcolor[rgb]{0.69,0.00,0.25}{##1}}}
\def\PY@tok@ow{\let\PY@bf=\textbf\def\PY@tc##1{\textcolor[rgb]{0.67,0.13,1.00}{##1}}}
\def\PY@tok@sb{\def\PY@tc##1{\textcolor[rgb]{0.73,0.13,0.13}{##1}}}
\def\PY@tok@k{\let\PY@bf=\textbf\def\PY@tc##1{\textcolor[rgb]{0.00,0.50,0.00}{##1}}}
\def\PY@tok@se{\let\PY@bf=\textbf\def\PY@tc##1{\textcolor[rgb]{0.73,0.40,0.13}{##1}}}
\def\PY@tok@sd{\let\PY@it=\textit\def\PY@tc##1{\textcolor[rgb]{0.73,0.13,0.13}{##1}}}

\def\PYZbs{\char`\\}
\def\PYZus{\char`\_}
\def\PYZob{\char`\{}
\def\PYZcb{\char`\}}
\def\PYZca{\char`\^}
% for compatibility with earlier versions
\def\PYZat{@}
\def\PYZlb{[}
\def\PYZrb{]}
\makeatother


\topmargin -1.0cm        % read Lamport p.163
\oddsidemargin -0.04cm   % read Lamport p.163
\evensidemargin -0.04cm  % same as oddsidemargin but for left-hand pages
\textwidth 16.59cm
\textheight 23.94cm 

%%\floatstyle{ruled}
\newfloat{example}{thp}{lop}
\floatname{example}{Example}


\title{HAWCNest: Design, Philosophy and Usage}
\author{John Pretz}
\date{Aug 23, 2010}
%\date{\today}
%\linenumbers
\begin{document}
\maketitle 

\abstract{HAWCNest is a C++ framework for configuring and arranging 
C++ components, termed 'services'.  This document describes the 
code from the perspective of somebody writing services, somebody
using services, and somebody configuring an application at the highest 
levels. The framework is written to be general enough for the different 
applications that will be used by the HAWC collaboration, most notably
the processing of event.}

\tableofcontents

\newpage

\section{Introduction}
HAWCNest is a framework intended for processing of HAWC events and 
subsequent analysis. It is currently a beta version suitable for developers 
and for first work on the HAWC software. 

The idea behind HAWCNest and other similar frameworks is to provide physicists 
with a uniform, coherent way to 'plug in' their code and make it usable by 
others. Code is broken down into discrete components that are added to the 
framework separately and in a regularized way. Since many people will 
contribute to the codebase, a framework is the most
straightforward way to combine pieces written by different people into the 
final applications. Individual physicists can
concentrate on writing their piece (perhaps a reconstruction, datareading
routine, calibration code, or database access code)
without worrying about the others. It should be noted that when a single 
person is writing a single program she is not
likely to bump into the problems that a framework is supposed to solve. It is 
mostly a tool for integrating pieces written by
many different people in a coherent way by prescribing how those pieces are 
written and combined together.

HAWCNest is modern C++, freely utilizing the C++ standard library and the 
boost libraries. It is thoroughly object oriented
and requires physicists to subscribe to object oriented
practice at the points where their code interacts with the framework.
No whining about how great Fortran was.

There are three sides or interfaces to HAWCNest. 
The first is what we might call the HAWCNest service-author interface. This is
the side of the code that allows people to extend the functionality and 
processing capability to the framework. Section \ref{saisection}
covers this interface and is primarily focused on the development of HAWCNest 
services. 
The second side of HAWCNest is the service-user interface, 
discussed in Section \ref{suisection}. This is a simple
interface
that allows people to utilize the code written by others, provided that that
code has been added into the framework already. 
The third side of HAWCNest is the
HAWCNest User interface. This is the aspect of the code that allows people to 
take and combine existing services and
components (written by people using the HAWCNest Developer interface) and 
reorganize them to form applications that
analyze data. The User interface is described in Section \ref{euisection}. 

Feedback on this project is appreciated and it is probably best to just email
the whole HAWC software list to get a healthy
discussion going about what we need in a framework.

The easiest way to become familiar with HAWCNest and how it's used is to 
have a small example.  Such an example is distributed with HAWCNest and more
information can be found on the HAWC wiki. This document includes several 
code snippits from that example and the example may be consulted for a 
functional example.  For completeness, the example is included as an appendix
to this document. Check the wiki for a more convenient downloadable 
version.

\section{Polymorphism in C++ and HAWCNest Services}

Object-oriented programming is a realatively recent model for programming.
It has gained ground in the physics community, particularly in C++.  Many
of the most valuable features of the approach are not commonly used in 
high-energy physics
applications.  This is probably due to the popularity of ROOT (which mis-uses
many features of C++) and the long history of Fortran and c in the physics 
community.  The feature that HAWCNest relies most on is the concept of 
run-time polymorphism (or inheiritance of virtual functions), 
and so this section reviews
the concept and it's proper usage.  This section assumes familiarity
with C++ and virtual functions.

The fundamental problem of software design is how to provide appropriate 
flexibility to change and re-use the software later.  
How many times have you written a 
program, gotten it working, and then wanted it to be changed in some 
important way or re-use just one part?  The various layers of indirection
that we put in programs allow re-use of parts, and flexibility in choosing
which of our code will run.  Highlighting this point, we have the 
famous 'fundamental theorem of computer science' as observed by computer
scientist David Wheeler, "All problems in computer science can be solved by 
another level of indirection."  

Example \ref{polymorphism} exhibits a simple C++ example which shows one 
way to approach this fundamental problem using C++ inheritance of 
virtual functions.  The pure virtual function {\tt DoSomething}
defined in the {\tt Interface} is not implemented in the base class.  
Neverless, the function {\tt something\_to\_do\_something} is able
to be compiled while completely ignorant of what what the instance of
{\tt Interface} is doing.  Indeed, as the {\tt main} shows, the function
actually does two different things, depending on which instance of 
{\tt Interface} is used.

\begin{example}
\begin{Verbatim}[commandchars=\\\{\}]
\PY{c+cp}{#}\PY{c+cp}{include <iostream>}

\PY{k}{using} \PY{k}{namespace} \PY{n}{std}\PY{p}{;}

\PY{k}{class} \PY{n+nc}{Interface}\PY{p}{\PYZob{}}
\PY{k}{public}\PY{o}{:}
  \PY{k}{virtual} \PY{k+kt}{void} \PY{n}{DoSomething}\PY{p}{(}\PY{p}{)} \PY{o}{=} \PY{l+m+mi}{0}\PY{p}{;}
\PY{p}{\PYZcb{}}\PY{p}{;}

\PY{k+kt}{void} \PY{n}{something\PYZus{}to\PYZus{}do\PYZus{}something}\PY{p}{(}\PY{n}{Interface}\PY{o}{*} \PY{n}{i}\PY{p}{)}\PY{p}{\PYZob{}}
  \PY{n}{i}\PY{o}{-}\PY{o}{>}\PY{n}{DoSomething}\PY{p}{(}\PY{p}{)}\PY{p}{;}
\PY{p}{\PYZcb{}}

\PY{k}{class} \PY{n+nc}{SomethingUseful} \PY{o}{:} \PY{k}{public} \PY{n}{Interface}\PY{p}{\PYZob{}}
\PY{k}{public}\PY{o}{:}
  \PY{k}{virtual} \PY{k+kt}{void} \PY{n}{DoSomething}\PY{p}{(}\PY{p}{)} \PY{p}{\PYZob{}} \PY{n}{cout}\PY{o}{<}\PY{o}{<}\PY{l+s}{"}\PY{l+s}{Something Useful....}\PY{l+s}{"}\PY{o}{<}\PY{o}{<}\PY{n}{endl}\PY{p}{;}\PY{p}{\PYZcb{}}
\PY{p}{\PYZcb{}}

\PY{k}{class} \PY{n+nc}{SomethingElseUseful} \PY{o}{:} \PY{k}{public} \PY{n}{Interface}\PY{p}{\PYZob{}}
\PY{k}{public}\PY{o}{:}
  \PY{k}{virtual} \PY{k+kt}{void} \PY{n}{DoSomething}\PY{p}{(}\PY{p}{)} \PY{p}{\PYZob{}} \PY{n}{cout}\PY{o}{<}\PY{o}{<}\PY{l+s}{"}\PY{l+s}{Something Else Useful....}\PY{l+s}{"}\PY{o}{<}\PY{o}{<}\PY{n}{endl}\PY{p}{;}\PY{p}{\PYZcb{}}
\PY{p}{\PYZcb{}}\PY{p}{;}

\PY{k+kt}{int} \PY{n}{main}\PY{p}{(}\PY{p}{)}\PY{p}{\PYZob{}}
  \PY{n}{Interface}\PY{o}{*} \PY{n}{i} \PY{o}{=} \PY{k}{new} \PY{n}{SomethingUseful}\PY{p}{(}\PY{p}{)}\PY{p}{;}
  \PY{n}{Interface}\PY{o}{*} \PY{n}{j} \PY{o}{=} \PY{k}{new} \PY{n}{SomethingElseUseful}\PY{p}{(}\PY{p}{)}\PY{p}{;}
  
  \PY{n}{something\PYZus{}to\PYZus{}do\PYZus{}something}\PY{p}{(}\PY{n}{i}\PY{p}{)}\PY{p}{;}
  \PY{n}{something\PYZus{}to\PYZus{}do\PYZus{}something}\PY{p}{(}\PY{n}{j}\PY{p}{)}\PY{p}{;}
  
\PY{p}{\PYZcb{}}
\end{Verbatim}

\caption{Example of straight polymorphism using vanilla C++}
\label{polymorphism}
\end{example}


This example captures the main use of C++ virtual functions, having
four important features.  

\begin{itemize}

\item In places where
we anticipate having multiple implementations or ways of doing things, 
an interface is specified in a base class 
(represented by the {\tt Interface} class in 
the example).

\item The various approaches are implemented via inheirtance
of the base class.

\item Code which calls this interface is ignorant of what
implementation or algorithm is used.  

\item The actual behavior is selected by choosing which of the implementations
of the interface are used.  

\end{itemize}

This is not the only way to provide flexibility in programming.  Many other
languages have constructs which aim to do the same job.  Even C++ has a wholly
different paradigm in templates (aka compile-time polymorphism).  But this
general approach to gaining flexibility and re-usability is what HAWCNest
makes use of, so the major themes of this example will feature 
prominently in the discussion of HAWCNest.

HAWCNest introduces the concept of a Service.  A Service is one unit of this
'fundamental problem'.  It is a piece of code for which we anticipate many
different (perhaps very different) solutions.  We will have many ways to 
reconstruct events, for example.  We could have many different algorithms
to assign random numbers.  We might have different formats for the data,
requiring different code to read from different data formats.  Each of these
would be a Service in the HAWCNest way of doing things.

So in the HAWCNest concept of the software, we will have dozens of Service
interfaces and implementations around for each unit of code.  Each of these
objects will be largely self-contained code units and each one will need
some configuration parameters.  The role of HAWCNest in this is to provide
a coherent way to organize these objects and provide a uniform way to 
configure and make implementations visible.  HAWCNest itself does not answer
the question of what services we will want.  It just provides a layer
for configuration and selection of services.

\section{Core Framework}

\subsection{HAWCNest Service-Author Interface}
\label{saisection}

Example \ref{service-declaration} shows minimal service.  The example is 
of a random number generator service.  Two classes are written.  The first
{\tt RandomNumberService} is the base service class which just defines the 
interface, and the implementation in {\tt STDRandomNumberService}.  
Code which uses the random number service will only know about the base 
class and no the implementation class, much like Example
\ref{polymorphism}.  

\begin{example}
\begin{Verbatim}[commandchars=\\\{\}]
\PY{c+cp}{#}\PY{c+cp}{include <hawcnest}\PY{c+cp}{/}\PY{c+cp}{Configuration.h>}

\PY{k}{class} \PY{n+nc}{RandomNumberService}
\PY{p}{\PYZob{}}
 \PY{k}{public}\PY{o}{:}
  \PY{k}{virtual} \PY{k+kt}{double} \PY{n}{Uniform}\PY{p}{(}\PY{k+kt}{double} \PY{n}{low}\PY{p}{,}\PY{k+kt}{double} \PY{n}{high}\PY{p}{)} \PY{o}{=} \PY{l+m+mi}{0}\PY{p}{;}

  \PY{k}{virtual} \PY{o}{~}\PY{n}{RandomNumberService}\PY{p}{(}\PY{p}{)}\PY{p}{\PYZob{}}\PY{p}{\PYZcb{}}
\PY{p}{\PYZcb{}}\PY{p}{;}

\PY{k}{class} \PY{n+nc}{STDRandomNumberService} \PY{o}{:} \PY{k}{public} \PY{n}{RandomNumberService}
\PY{p}{\PYZob{}}
 \PY{k}{public}\PY{o}{:}
  \PY{k}{typedef} \PY{n}{RandomNumberService} \PY{n}{Interface}\PY{p}{;}

  \PY{n}{Configuration} \PY{n}{DefaultConfiguration}\PY{p}{(}\PY{p}{)}\PY{p}{\PYZob{}}
    \PY{n}{Configuration} \PY{n}{config}\PY{p}{;}
    \PY{n}{config}\PY{p}{.}\PY{n}{Parameter}\PY{o}{<}\PY{k+kt}{int}\PY{o}{>}\PY{p}{(}\PY{l+s}{"}\PY{l+s}{Seed}\PY{l+s}{"}\PY{p}{,}\PY{l+m+mi}{0}\PY{p}{)}\PY{p}{;}
    \PY{k}{return} \PY{n}{config}\PY{p}{;}
  \PY{p}{\PYZcb{}}

  \PY{k+kt}{void} \PY{n}{Initialize}\PY{p}{(}\PY{k}{const} \PY{n}{Configuration}\PY{o}{&} \PY{n}{config}\PY{p}{)}\PY{p}{\PYZob{}}
    \PY{n}{config}\PY{p}{.}\PY{n}{GetParameter}\PY{p}{(}\PY{l+s}{"}\PY{l+s}{Seed}\PY{l+s}{"}\PY{p}{,}\PY{n}{seed\PYZus{}}\PY{p}{)}\PY{p}{;}
    \PY{n}{printf}\PY{p}{(}\PY{l+s}{"}\PY{l+s}{seeding random number generator with seed %d}\PY{l+s+se}{\PYZbs{}n}\PY{l+s}{"}\PY{p}{,}\PY{n}{seed\PYZus{}}\PY{p}{)}\PY{p}{;}
    \PY{n}{srand}\PY{p}{(}\PY{n}{seed\PYZus{}}\PY{p}{)}\PY{p}{;}
  \PY{p}{\PYZcb{}}

  \PY{k+kt}{double} \PY{n}{Uniform}\PY{p}{(}\PY{k+kt}{double} \PY{n}{low}\PY{p}{,}\PY{k+kt}{double} \PY{n}{high}\PY{p}{)}\PY{p}{\PYZob{}}
    \PY{k+kt}{double} \PY{n}{rnd} \PY{o}{=} \PY{p}{(}\PY{k+kt}{double}\PY{p}{)}\PY{n}{rand}\PY{p}{(}\PY{p}{)} \PY{o}{/}\PY{p}{(}\PY{k+kt}{double}\PY{p}{)}\PY{n}{RAND\PYZus{}MAX}\PY{p}{;}
    \PY{k+kt}{double} \PY{n}{toReturn} \PY{o}{=} \PY{n}{rnd} \PY{o}{*}\PY{p}{(}\PY{n}{high} \PY{o}{-} \PY{n}{low}\PY{p}{)} \PY{o}{+} \PY{n}{low}\PY{p}{;}
    \PY{k}{return} \PY{n}{toReturn}\PY{p}{;}
  \PY{p}{\PYZcb{}}

 \PY{k}{private}\PY{o}{:}
  \PY{k+kt}{int} \PY{n}{seed\PYZus{}}\PY{p}{;}
\PY{p}{\PYZcb{}}\PY{p}{;}
\end{Verbatim}

\caption{Example declaration of a service.}
\label{service-declaration}
\end{example}

The {\tt STDRandomNumberService} class has a number of important 
features for integrating it into the framework.  

The first important feature is the {\tt Interface} typedef.  That type
should be the way that people will access this service.  Note that it does
not have to be the inherited class.  It can be a base class higher up an
inheritance tree or the class type itself if there is no base class, but 
the anticipated usage is that it will be a pure virtual
base class for the service.

Second is the 
method {\tt Configuration DefaultConfiguration()}.  
%The {\tt Configuration} class is documented in Appendix \ref{configclass}.
This method is 
used by the service to specify what configuration parameters the 
service instance will take.  This service takes one parameter, a seed value.
Note that the parameter type is specified ({\tt int}) along with a name
({\tt "Seed" }) and a default value ({\tt 0}).  The default value is
optional.  If no default value is specified the framework will generate
an exception if no value is specified when the framework is configured.
The framework supports six types of parameters: {\tt int}, {\tt double},
{\tt string}, {\tt vector<int>}, {\tt vector<double>} and
{\tt vector<string>}.

Additionally the {\tt void Initialize(const Configuration\& config) }
method which is the method that is called by the framework after the 
framework user has chosen all the configuration parameters.  Note the usage
of the {\tt Configuration} class to retrieve the parameters after they've been
set by the user.  The {\tt GetParameter} method is templated and determines
the type of retrieved parameter by the type of the passed-in reference.
Note that it is during the {\tt Initialize} method that we do start-up
calculations and, in this case, set the random number seed.

Finally, we have the actual implemented method, {\tt Uniform}.  This will
naturally differ from service to service.  

Finally, we should note that this example shows one implementation of 
{\tt RandomNumberService} which is {\tt STDRandomNumberService}.  There
is nothing that prevents multiple implementations and, in fact, the 
framework really is intended to have multiple implementations of 
service base classes.  A second implementation, for instance, might use
ROOT's {\tt TRandom} service instead of the standard library like 
{\tt STDRandomNumberService} does.  
If there is more than one implementation of a service interface, the
end-user selects which implementation will be used.  See Section 
\ref{euisection}. It is possible
for many implementations to be used simultaneously within the framework.

\subsection{HAWCNest Service-User Interface}
\label{suisection}

The service-user interface is very simple and is exhibited in 
Example \ref{getting-service}.  It amounts to one function call, 
{\tt GetService< >}.  The type that goes into the template is the 
type of the base service class that is being retrieved, in this case
{\tt RandomNumberService}.  The other thing that is needed to retrieve a service
is a string which is the name of the service instance.  Note that
in this case the service name is "rand".  In most cases the name of the service
that is being retrieved would be itself configured by a parameter so that
users can change the instance of the service used.  After retrieving a
reference to the base service, the user can call the virtual base class
methods without knowing the full type of the instance.  That is, without
knowing whether this is a {\tt STDRandomNumberService} or some other
implementation.

\begin{example}
\begin{Verbatim}[commandchars=\\\{\}]
\PY{c+cp}{#}\PY{c+cp}{include <hawcnest}\PY{c+cp}{/}\PY{c+cp}{Service.h>}

\PY{n}{RandomNumberService}\PY{o}{&} \PY{n}{random} \PY{o}{=} \PY{n}{GetService}\PY{o}{<}\PY{n}{RandomNumberService}\PY{o}{>}\PY{p}{(}\PY{l+s}{"}\PY{l+s}{rand}\PY{l+s}{"}\PY{p}{)}\PY{p}{;}

\PY{n}{CoreReconstructionPtr} \PY{n}{core}\PY{p}{(}\PY{k}{new} \PY{n}{CoreReconstruction}\PY{p}{(}\PY{p}{)}\PY{p}{)}\PY{p}{;}
\PY{n}{core}\PY{o}{-}\PY{o}{>}\PY{n}{corex} \PY{o}{=} \PY{n}{random}\PY{p}{.}\PY{n}{Uniform}\PY{p}{(}\PY{o}{-}\PY{l+m+mi}{2000}\PY{p}{,}\PY{l+m+mi}{2000}\PY{p}{)}\PY{p}{;}
\PY{n}{core}\PY{o}{-}\PY{o}{>}\PY{n}{corex} \PY{o}{=} \PY{n}{random}\PY{p}{.}\PY{n}{Uniform}\PY{p}{(}\PY{o}{-}\PY{l+m+mi}{2000}\PY{p}{,}\PY{l+m+mi}{2000}\PY{p}{)}\PY{p}{;}
\end{Verbatim}

\caption{Getting a Service}
\label{getting-service}
\end{example}



\subsection{HAWCNest End-User Interface}
\label{euisection}

The HAWCNest end-user interface is the side of the code that is used to 
configure and run a HAWCNest application.  This is the interface where
we choose what services we will use and what their configuration parameters
are.  The interface is pretty simple and is exhibited in 
Example \ref{configuring-hawcnest}.  This side of the code is accessed through
a class HAWCNest.  This class has methods to add and configure services.
The code prevents more than one instance of HAWCNest at a time, and it is
anticipated that this class is used within the {\tt main} of the program.

\begin{example}
\begin{Verbatim}[commandchars=\\\{\}]
\PY{c+cp}{#}\PY{c+cp}{include <hawcnest}\PY{c+cp}{/}\PY{c+cp}{HAWCNest.h>}
\PY{c+cp}{#}\PY{c+cp}{include "STDRandomNumberSerice.h"}
\PY{c+cp}{#}\PY{c+cp}{include "DBCalibrationService.h"}

\PY{n}{HAWCNest} \PY{n}{nest}\PY{p}{;}

\PY{n}{nest}\PY{p}{.}\PY{n}{Service}\PY{o}{<}\PY{n}{STDRandomNumberService}\PY{o}{>}\PY{p}{(}\PY{l+s}{"}\PY{l+s}{rand}\PY{l+s}{"}\PY{p}{)}
  \PY{p}{(}\PY{l+s}{"}\PY{l+s}{Seed}\PY{l+s}{"}\PY{p}{,}\PY{l+m+mi}{12345}\PY{p}{)}\PY{p}{;}

\PY{n}{nest}\PY{p}{.}\PY{n}{Service}\PY{o}{<}\PY{n}{DBCalibrationService}\PY{o}{>}\PY{p}{(}\PY{l+s}{"}\PY{l+s}{calib}\PY{l+s}{"}\PY{p}{)}
  \PY{p}{(}\PY{l+s}{"}\PY{l+s}{server}\PY{l+s}{"}\PY{p}{,}\PY{l+s}{"}\PY{l+s}{mildb.umd.edu}\PY{l+s}{"}\PY{p}{)}
  \PY{p}{(}\PY{l+s}{"}\PY{l+s}{uname}\PY{l+s}{"}\PY{p}{,}\PY{l+s}{"}\PY{l+s}{milagro}\PY{l+s}{"}\PY{p}{)}
  \PY{p}{(}\PY{l+s}{"}\PY{l+s}{password}\PY{l+s}{"}\PY{p}{,}\PY{l+s}{"}\PY{l+s}{secret}\PY{l+s}{"}\PY{p}{)}\PY{p}{;}

\PY{n}{nest}\PY{p}{.}\PY{n}{Configure}\PY{p}{(}\PY{p}{)}\PY{p}{;}
\end{Verbatim}

\caption{A Snippet exhibiting how HAWCNest is configured.}
\label{configuring-hawcnest}
\end{example}

This interface is straightforward.  An instance of {\tt HAWCNest} is created 
and then
choices of services are made via {\tt Service< >}.  
The parameters of the service implementation can be made at this time.
Here, the full type
of the service is chosen and a name is assigned.  This name should match
the name that is used to retrieve the service in Example 
\ref{getting-service}.

\section{Event Processing Model}
\label{processingsection}

The previous sections have outlined the major functionality of HAWCNest. It
is essentially a broker for multiple implementation of base services. The 
framework is agnostic on what those services do. There is one set of service
interfaces that is distributed with the framework for the purpose of processing
HAWC events.  In some strict since they are not part of the framework 
proper but are just distributed with it since the primary use of HAWCNest
is likely to be processing HAWC events. 

The following sections will describe the major classes and how they 
work together to form the processing model for events.

\subsection{Bag}

The {\tt Bag} is the container for event data. The {\tt Bag} is a generic data
container in which data of any type which inheirits from {\tt Baggable} can
be placed. Data is stored in an instance of {\tt Bag} with a {\tt string}
key much like a standard library {\tt map}. 

The {\tt Bag} is a add-only data structure. New data members can be added
dynamically, but once they are added they cannot be retrieved except by 
{\tt const} reference or pointer and are not modifiable. The idea is that the
members of the {\tt Bag} instance representing this event grows as 
processing steps add data. The add-only nature of the {\tt Bag} is in order
that the output of previous processing steps are preserved.

Example \ref{bagexample} shows typical usage of the Bag. It is a very 
straight-forward interface. The example shows the definition of some 
toy {\tt TestData} class. This is a stand-in for the anticipated classes
and structures
that would make up the event data in memory. For instance, there might be 
classes for storing Monte-Carlo information, the simulated 
event signal and reconstructed event information.

\begin{example}
\begin{Verbatim}[commandchars=\\\{\}]
\PY{c+c1}{// declaring some 'data'}
\PY{c+c1}{// must inherit from Baggable}
\PY{k}{class} \PY{n+nc}{TestData} \PY{o}{:} \PY{k}{public} \PY{n}{Baggable}\PY{p}{\PYZob{}}
\PY{p}{\PYZcb{}}\PY{p}{;}

\PY{c+c1}{// these typedefs are just for convenience sake}
\PY{k}{typedef} \PY{n}{boost}\PY{o}{:}\PY{o}{:}\PY{n}{shared\PYZus{}ptr}\PY{o}{<}\PY{n}{TestData}\PY{o}{>} \PY{n}{TestDataPtr}\PY{p}{;}
\PY{k}{typedef} \PY{n}{boost}\PY{o}{:}\PY{o}{:}\PY{n}{shared\PYZus{}ptr}\PY{o}{<}\PY{k}{const} \PY{n}{TestData}\PY{o}{>} \PY{n}{TestDataConstPtr}\PY{p}{;}

\PY{c+c1}{// exhibiting usage}
\PY{n}{Bag} \PY{n}{bag}\PY{p}{;}

\PY{n}{TestDataPtr} \PY{n}{testdata}\PY{p}{(}\PY{k}{new} \PY{n}{TestData}\PY{p}{(}\PY{p}{)}\PY{p}{)}\PY{p}{;}
\PY{n}{bag}\PY{p}{.}\PY{n}{Put}\PY{p}{(}\PY{l+s}{"}\PY{l+s}{testdata}\PY{l+s}{"}\PY{p}{,}\PY{n}{testdata}\PY{p}{)}\PY{p}{;}

\PY{c+c1}{// Getting the data out by shared ptr.  Returns an empty pointer if the }
\PY{c+c1}{//object type mismatches or if the object doesn't exist}
\PY{n}{TestDataConstPtr} \PY{n}{testdata\PYZus{}out} \PY{o}{=} \PY{n}{bag}\PY{p}{.}\PY{n}{Get}\PY{o}{<}\PY{n}{TestDataConstPtr}\PY{o}{>}\PY{p}{(}\PY{l+s}{"}\PY{l+s}{testdata}\PY{l+s}{"}\PY{p}{)}\PY{p}{;}

\PY{c+c1}{// Getting object by reference.  Throws bag\PYZus{}exception if the object type}
\PY{c+c1}{// doesn't match or if the object doesn't exist.}
\PY{k}{const} \PY{n}{TestData}\PY{o}{&} \PY{n}{testdata\PYZus{}out\PYZus{}ref} \PY{o}{=} \PY{n}{bag}\PY{p}{.}\PY{n}{Get}\PY{o}{<}\PY{n}{TestData}\PY{o}{>}\PY{p}{(}\PY{l+s}{"}\PY{l+s}{testdata}\PY{l+s}{"}\PY{p}{)}\PY{p}{;}

\PY{c+c1}{// operator<< is defined for some easy pretty-printing}
\PY{n}{cout}\PY{o}{<}\PY{o}{<}\PY{n}{bag}\PY{o}{<}\PY{o}{<}\PY{n}{endl}\PY{p}{;}
\end{Verbatim}

\caption{Example usage of the {\tt Bag} structure.}
\label{bagexample}
\end{example}

%Comments on boost::shared\_ptr;

\subsection{Modules and Sources}

A {\tt Module} and {\tt Source} are two interfaces that are used to perform the
processing steps.  Implementations of these interfaces are added
to the framework in the same way as regular HAWCNest services.  Appendix
\ref{hawcnest-example-appendix}
exhibits a complete example of a processing chain. In the case of a 
{\tt Source}, the user implements {\tt BagPtr Source::Next()}. The 
{\tt Source} is the thing that begins the event processing chain by reading
an event from a file or generating it in some way.  It is expected to allocate
a {\tt new Bag}, filling it with some data.  It should return a {\tt NULL} 
pointer when there are no more events.

The {\tt Module} interface is for subsequent steps in the processing chain.
The only method that subclasses have to implement is 
{\tt Module::Result Module::Process(BagPtr)} which should be the Module's 
manipulation of the {\tt Bag}, reading the data it needs and writing out new
data that it has generated.

\subsection{Main Loop}

The {\tt MainLoop} is the final piece of the processing model.  The 
{\tt MainLoop} is a very simple service with one method {\tt void Execute()}.
There is one implementation currently {\tt SequentialMainLoop} which
takes configuration parameters of the name of the {\tt Source} and a list 
of {\tt Module}s to process.  The {\tt SequentialMainLoop} gets the 
steps in the processing chain by 
{\tt GetService<>} and calls
{\tt Process} on each one until the {\tt Source} stops returning data. 
Depending on the {\tt Module::Result} returned, the main loop may stop 
processing on an event all-together.

The important thing to note here is that the processing model is independent
of the framework proper. Different implementations of {\tt MainLoop} can
generate more complex topologies or behavior. The most useful of these should
be incorporated back into the framework. Indeed, a {\tt MainLoop} need not
be used at all. It is perfectly conceivable to call each of the {\tt Module}
and {\tt Source} modules directly. The user has total flexibility over how
the processing chain operates and can choose a level to hook in that is 
appropriate for her application.

\section{Python Interface}

Distributed with HAWCNest is a python interface to the core framework. This
interface is only used for configuring the framework and it is not possible to
implement or use services from python. It is conceivable to implement this 
someday, but at the moment the python can only configure compiled C++.

The python interface is essentially one class {\tt HAWCNest}, and one
function {\tt load} which are found in the {\tt HAWCNest} package. The
{\tt load} function should be used to load shared libraries with the compiled
services you need, and the usage of the {\tt HAWCNest} is nearly identical
to the C++ version, except for the use of python syntax. See 
Example \ref{python-example} for the usage.

\begin{example}
\begin{Verbatim}[commandchars=\\\{\}]
\PY{c}{# importing the needed classes and functions}
\PY{k+kn}{from} \PY{n+nn}{HAWCNest} \PY{k+kn}{import} \PY{n}{HAWCNest}
\PY{k+kn}{from} \PY{n+nn}{HAWCNest} \PY{k+kn}{import} \PY{n}{load}
\PY{k+kn}{from} \PY{n+nn}{HAWCNest} \PY{k+kn}{import} \PY{n}{GetService\PYZus{}MainLoop}

\PY{c}{# loading shared libraries that might be needed for processing}
\PY{n}{load}\PY{p}{(}\PY{l+s}{"}\PY{l+s}{libhawcnest}\PY{l+s}{"}\PY{p}{)}
\PY{n}{load}\PY{p}{(}\PY{l+s}{"}\PY{l+s}{libexample-hawcnest}\PY{l+s}{"}\PY{p}{)}

\PY{c}{# This is nearly identical to the C++ interface, except it uses python}
\PY{c}{# syntax}
\PY{n}{nest} \PY{o}{=} \PY{n}{HAWCNest}\PY{p}{(}\PY{p}{)}

\PY{c}{# adding and configuring services}
\PY{n}{nest}\PY{o}{.}\PY{n}{Service}\PY{p}{(}\PY{l+s}{"}\PY{l+s}{STDRandomNumberService}\PY{l+s}{"}\PY{p}{,}\PY{l+s}{"}\PY{l+s}{rand}\PY{l+s}{"}\PY{p}{)}\PY{p}{(}
    \PY{p}{(}\PY{l+s}{"}\PY{l+s}{Seed}\PY{l+s}{"}\PY{p}{,}\PY{l+m+mi}{12345}\PY{p}{)}
    \PY{p}{)}
\PY{n}{nest}\PY{o}{.}\PY{n}{Service}\PY{p}{(}\PY{l+s}{"}\PY{l+s}{DBCalibrationService}\PY{l+s}{"}\PY{p}{,}\PY{l+s}{"}\PY{l+s}{calib}\PY{l+s}{"}\PY{p}{)}\PY{p}{(}
    \PY{p}{(}\PY{l+s}{"}\PY{l+s}{server}\PY{l+s}{"}\PY{p}{,}\PY{l+s}{"}\PY{l+s}{mildb.umd.edu}\PY{l+s}{"}\PY{p}{)}
    \PY{p}{(}\PY{l+s}{"}\PY{l+s}{uname}\PY{l+s}{"}\PY{p}{,}\PY{l+s}{"}\PY{l+s}{milagro}\PY{l+s}{"}\PY{p}{)}
    \PY{p}{(}\PY{l+s}{"}\PY{l+s}{password}\PY{l+s}{"}\PY{p}{,}\PY{l+s}{"}\PY{l+s}{topsecret}\PY{l+s}{"}\PY{p}{)}
    \PY{p}{)}
\PY{n}{nest}\PY{o}{.}\PY{n}{Service}\PY{p}{(}\PY{l+s}{"}\PY{l+s}{ExampleSource}\PY{l+s}{"}\PY{p}{,}\PY{l+s}{"}\PY{l+s}{source}\PY{l+s}{"}\PY{p}{)}\PY{p}{(}
    \PY{p}{(}\PY{l+s}{"}\PY{l+s}{input}\PY{l+s}{"}\PY{p}{,}\PY{l+s}{"}\PY{l+s}{myinputfile.dat}\PY{l+s}{"}\PY{p}{)}
    \PY{p}{(}\PY{l+s}{"}\PY{l+s}{maxevents}\PY{l+s}{"}\PY{p}{,}\PY{l+m+mi}{20}\PY{p}{)}
    \PY{p}{)}
\PY{n}{nest}\PY{o}{.}\PY{n}{Service}\PY{p}{(}\PY{l+s}{"}\PY{l+s}{CalibrationModule}\PY{l+s}{"}\PY{p}{,}\PY{l+s}{"}\PY{l+s}{calibmodule}\PY{l+s}{"}\PY{p}{)}
\PY{n}{nest}\PY{o}{.}\PY{n}{Service}\PY{p}{(}\PY{l+s}{"}\PY{l+s}{ReconstructionModule\PYZus{}COM}\PY{l+s}{"}\PY{p}{,}\PY{l+s}{"}\PY{l+s}{reco\PYZus{}com}\PY{l+s}{"}\PY{p}{)}
\PY{n}{nest}\PY{o}{.}\PY{n}{Service}\PY{p}{(}\PY{l+s}{"}\PY{l+s}{ReconstructionModule\PYZus{}Gauss}\PY{l+s}{"}\PY{p}{,}\PY{l+s}{"}\PY{l+s}{reco\PYZus{}gauss}\PY{l+s}{"}\PY{p}{)}
\PY{n}{nest}\PY{o}{.}\PY{n}{Service}\PY{p}{(}\PY{l+s}{"}\PY{l+s}{PrintingModule}\PY{l+s}{"}\PY{p}{,}\PY{l+s}{"}\PY{l+s}{print}\PY{l+s}{"}\PY{p}{)}
\PY{n}{nest}\PY{o}{.}\PY{n}{Service}\PY{p}{(}\PY{l+s}{"}\PY{l+s}{SequentialMainLoop}\PY{l+s}{"}\PY{p}{,}\PY{l+s}{"}\PY{l+s}{mainloop}\PY{l+s}{"}\PY{p}{)}\PY{p}{(}
    \PY{p}{(}\PY{l+s}{"}\PY{l+s}{source}\PY{l+s}{"}\PY{p}{,}\PY{l+s}{"}\PY{l+s}{source}\PY{l+s}{"}\PY{p}{)}
    \PY{p}{(}\PY{l+s}{"}\PY{l+s}{modulechain}\PY{l+s}{"}\PY{p}{,}\PY{p}{[}\PY{l+s}{"}\PY{l+s}{print}\PY{l+s}{"}\PY{p}{,}\PY{l+s}{"}\PY{l+s}{calibmodule}\PY{l+s}{"}\PY{p}{,}\PY{l+s}{"}\PY{l+s}{reco\PYZus{}com}\PY{l+s}{"}\PY{p}{,}
                    \PY{l+s}{"}\PY{l+s}{reco\PYZus{}gauss}\PY{l+s}{"}\PY{p}{,}\PY{l+s}{"}\PY{l+s}{print}\PY{l+s}{"}\PY{p}{,}\PY{l+s}{"}\PY{l+s}{mymodule}\PY{l+s}{"}\PY{p}{]}\PY{p}{)}\PY{p}{,}
    \PY{p}{)}
\PY{n}{nest}\PY{o}{.}\PY{n}{Configure}\PY{p}{(}\PY{p}{)}
\PY{n}{main} \PY{o}{=} \PY{n}{GetService\PYZus{}MainLoop}\PY{p}{(}\PY{l+s}{"}\PY{l+s}{mainloop}\PY{l+s}{"}\PY{p}{)}
\PY{n}{loop}\PY{o}{.}\PY{n}{Execute}\PY{p}{(}\PY{p}{)}
\PY{n}{nest}\PY{o}{.}\PY{n}{Finish}\PY{p}{(}\PY{p}{)}
\end{Verbatim}

\caption{Example usage of the python interface.}
\label{python-example}
\end{example}

\appendix


%% \section{User Questions and Answers}

%% This area contains the answers to some perhaps-puzzling or counter-intuitive
%% aspects of the framework. 

%% \subsection{Why no base class for services?}

%% \subsection{What's the difference between a Module and a Service?}

%% \subsection{Globals}

\section{Logging Interface}

The logging system is included in the 
header {\tt <hawcnest/Logging.h>}.
The logging system in HAWCNest is not complete. The message-generator
side is complete. There are functions for dumping logging messages. The 
side of the logging system that directs the generated messages
is not complete.  Generated messages are just dumped to the console, though
the threshold to actually print a message is configurable.  See Example 
\ref{logging-example} for example usage.

The message-generator side of the code is six macros, detailed below.
The macros take standard {\tt printf} syntax, but there is no need
to append the trailing {\tt \textbackslash n}.  The guide below gives 
rough guidance on how the logging levels should be interpreted.

\begin{itemize}

\item {\tt log\_fatal} - Only called for fatal errors. Expected that a throw or exit is forthcoming.

\item {\tt log\_error} - Non-fatal (recoverable) exception. No exception thrown

\item {\tt log\_warn} - Possible error conditions approaching....

\item {\tt log\_info} - Information to tell operator what's going on.  Fewer 
than 
one message per processed event. A user should feel comfortable turning on 
the INFO level without fear of flooding the console.

\item log\_debug - Information for system expert. Approximately one message 
per component per event is expected.

\item log\_trace - Chornic logorrhea. For step by step debugging. These calls
can be used liberally, perhaps even in place of comments.

\end{itemize}

The end user can alter the level of the messages as seen in Example 
\ref{logging-example}.  Note that if the code is compiled with {\tt -DNDEBUG}
then trace, debug, and info statements are completely compiled out
resulting in 0 runtime cost.  

\begin{example}
\begin{Verbatim}[commandchars=\\\{\}]
\PY{c+c1}{// The Logging header}
\PY{c+cp}{#}\PY{c+cp}{include <hawcnest}\PY{c+cp}{/}\PY{c+cp}{Logging.h>}

\PY{c+c1}{// Exhibiting how to set the logging threshold to TRACE}
\PY{c+c1}{// This should only be done in end-user code and not in }
\PY{c+c1}{// Modules or Services}
\PY{n}{LoggingConfig}\PY{o}{:}\PY{o}{:}\PY{n}{Instance}\PY{p}{(}\PY{p}{)}\PY{p}{.}\PY{n}{SetDefaultLogLevel}\PY{p}{(}\PY{n}{TRACE}\PY{p}{)}\PY{p}{;}

\PY{n}{log\PYZus{}trace}\PY{p}{(}\PY{l+s}{"}\PY{l+s}{An example at TRACE level. Note that there is no trailing newline}\PY{l+s}{"}\PY{p}{)}\PY{p}{;}
\PY{n}{log\PYZus{}trace}\PY{p}{(}\PY{l+s}{"}\PY{l+s}{Since log\PYZus{}trace statements are compiled out at high optimization }\PY{l+s}{"}
          \PY{l+s}{"}\PY{l+s}{levels, you can use them as an alternative to comments where it is }\PY{l+s}{"}
          \PY{l+s}{"}\PY{l+s}{appropriate}\PY{l+s}{"}\PY{p}{)}\PY{p}{;}

\PY{k+kt}{int} \PY{n}{i} \PY{o}{=} \PY{l+m+mi}{5}\PY{p}{;}
\PY{n}{log\PYZus{}warn}\PY{p}{(}\PY{l+s}{"}\PY{l+s}{An example at WARN level. Note that we can use standard printf }\PY{l+s}{"}
         \PY{l+s}{"}\PY{l+s}{syntax to print %d}\PY{l+s}{"}\PY{p}{,}\PY{n}{i}\PY{p}{)}\PY{p}{;}

\PY{n}{log\PYZus{}fatal}\PY{p}{(}\PY{l+s}{"}\PY{l+s}{Example FATAL call. NB that the call throws an exception}\PY{l+s}{"}\PY{p}{)}\PY{p}{;}
\end{Verbatim}

\caption{A Snippet exhibiting how to use and configure the logging interface.}
\label{logging-example}
\end{example}

%
%\section{Configuration Class Documentation}
%\label{configclass}
%
%\section{HAWCNest Class Documentation}
%
%\section{GetService Documentation}

\newpage
\section{HAWCNest Example Application}
\label{hawcnest-example-appendix}

\begin{Verbatim}[commandchars=\\\{\}]
\PY{c+cm}{/**}
\PY{c+cm}{ * FILE ExampleComponents.h}
\PY{c+cm}{ */}

\PY{c+cp}{#}\PY{c+cp}{ifndef EXAMPLE\PYZus{}MODULES\PYZus{}H}
\PY{c+cp}{#}\PY{c+cp}{define EXAMPLE\PYZus{}MODULES\PYZus{}H}

\PY{c+cp}{#}\PY{c+cp}{include <cstdio>}
\PY{c+cp}{#}\PY{c+cp}{include <hawcnest}\PY{c+cp}{/}\PY{c+cp}{processing}\PY{c+cp}{/}\PY{c+cp}{Module.h>}
\PY{c+cp}{#}\PY{c+cp}{include <hawcnest}\PY{c+cp}{/}\PY{c+cp}{Service.h>}
\PY{c+cp}{#}\PY{c+cp}{include <hawcnest}\PY{c+cp}{/}\PY{c+cp}{processing}\PY{c+cp}{/}\PY{c+cp}{Source.h>}

\PY{c+c1}{// some structures for the bag}
\PY{k}{struct} \PY{n}{EventHeader} \PY{o}{:} \PY{n}{public} \PY{n}{Baggable}\PY{p}{\PYZob{}}
  \PY{k+kt}{int} \PY{n}{event\PYZus{}number}\PY{p}{;}
  \PY{k+kt}{int} \PY{n}{run\PYZus{}number}\PY{p}{;}
  \PY{k+kt}{int} \PY{n}{hit\PYZus{}channels}\PY{p}{;}
\PY{p}{\PYZcb{}}\PY{p}{;}

\PY{k}{typedef} \PY{n}{boost}\PY{o}{:}\PY{o}{:}\PY{n}{shared\PYZus{}ptr}\PY{o}{<}\PY{n}{EventHeader}\PY{o}{>} \PY{n}{EventHeaderPtr}\PY{p}{;}
\PY{k}{typedef} \PY{n}{boost}\PY{o}{:}\PY{o}{:}\PY{n}{shared\PYZus{}ptr}\PY{o}{<}\PY{k}{const} \PY{n}{EventHeader}\PY{o}{>} \PY{n}{EventHeaderConstPtr}\PY{p}{;}

\PY{k}{struct} \PY{n}{HitData} \PY{o}{:} \PY{n}{public} \PY{n}{Baggable}\PY{p}{\PYZob{}}
  \PY{n}{std}\PY{o}{:}\PY{o}{:}\PY{n}{vector}\PY{o}{<}\PY{k+kt}{int}\PY{o}{>} \PY{n}{hit\PYZus{}channels}\PY{p}{;}
  \PY{n}{std}\PY{o}{:}\PY{o}{:}\PY{n}{vector}\PY{o}{<}\PY{k+kt}{float}\PY{o}{>} \PY{n}{hit\PYZus{}times}\PY{p}{;}
\PY{p}{\PYZcb{}}\PY{p}{;}

\PY{k}{typedef} \PY{n}{boost}\PY{o}{:}\PY{o}{:}\PY{n}{shared\PYZus{}ptr}\PY{o}{<}\PY{n}{HitData}\PY{o}{>} \PY{n}{HitDataPtr}\PY{p}{;}
\PY{k}{typedef} \PY{n}{boost}\PY{o}{:}\PY{o}{:}\PY{n}{shared\PYZus{}ptr}\PY{o}{<}\PY{k}{const} \PY{n}{HitData}\PY{o}{>} \PY{n}{HitDataConstPtr}\PY{p}{;}

\PY{k}{struct} \PY{n}{CoreReconstruction} \PY{o}{:} \PY{n}{public} \PY{n}{Baggable}\PY{p}{\PYZob{}}
  \PY{k+kt}{float} \PY{n}{corex}\PY{p}{;}
  \PY{k+kt}{float} \PY{n}{corey}\PY{p}{;}
\PY{p}{\PYZcb{}}\PY{p}{;}

\PY{k}{typedef} \PY{n}{boost}\PY{o}{:}\PY{o}{:}\PY{n}{shared\PYZus{}ptr}\PY{o}{<}\PY{n}{CoreReconstruction}\PY{o}{>} \PY{n}{CoreReconstructionPtr}\PY{p}{;}
\PY{k}{typedef} \PY{n}{boost}\PY{o}{:}\PY{o}{:}\PY{n}{shared\PYZus{}ptr}\PY{o}{<}\PY{k}{const} \PY{n}{CoreReconstruction}\PY{o}{>} \PY{n}{CoreReconstructionConstPtr}\PY{p}{;}

\PY{c+c1}{// In this simple example, I have implemented all of the components (services}
\PY{c+c1}{// and modules) used in this example.  For each service, I have two }
\PY{c+c1}{// implementations}

\PY{c+c1}{// First are the services.  The services come in at least two parts.  The first}
\PY{c+c1}{// is the interface.  This is a pure virtual base class which just defines}
\PY{c+c1}{// what it is that the service provides.  The pure virtual base class}
\PY{c+c1}{// is akin to the Java 'Interface' construct.  The second part is the}
\PY{c+c1}{// actual implementation which inherits from the interface and implements}
\PY{c+c1}{// the methods defined in the interface as well as the methods required}
\PY{c+c1}{// of the framework.}

\PY{c+c1}{// The interface for the RandomNumberService.  Note that the client code}
\PY{c+c1}{// calls RandomNumberService& random = GetService<RandomNumberService()}
\PY{c+c1}{// So this is the only thing that they know about. }
\PY{n}{class} \PY{n}{RandomNumberService}
\PY{p}{\PYZob{}}
 \PY{n+nl}{public:}
  \PY{c+c1}{// give a number drawn from a uniform distribution}
  \PY{k}{virtual} \PY{k+kt}{double} \PY{n}{Uniform}\PY{p}{(}\PY{k+kt}{double} \PY{n}{low}\PY{p}{,}\PY{k+kt}{double} \PY{n}{high}\PY{p}{)} \PY{o}{=} \PY{l+m+mi}{0}\PY{p}{;}
  
  \PY{k}{virtual} \PY{o}{~}\PY{n}{RandomNumberService}\PY{p}{(}\PY{p}{)}\PY{p}{\PYZob{}}\PY{p}{\PYZcb{}}
\PY{p}{\PYZcb{}}\PY{p}{;}

\PY{c+c1}{// The first implementation of this service uses std::rand }
\PY{n}{class} \PY{n}{STDRandomNumberService} \PY{o}{:} \PY{n}{public} \PY{n}{RandomNumberService}
\PY{p}{\PYZob{}}
 \PY{n+nl}{public:}
  \PY{c+c1}{// This typedef tells the framework that this class will be used}
  \PY{c+c1}{// to satisfy the RandomNumberService interface}
  \PY{k}{typedef} \PY{n}{RandomNumberService} \PY{n}{Interface}\PY{p}{;}

  \PY{c+c1}{// This method is where you declare your default configuration}
  \PY{c+c1}{// and all the configuration parameters that this component will take.}
  \PY{c+c1}{// This random number generator takes one parameter, a 'seed'.}
  \PY{n}{Configuration} \PY{n}{DefaultConfiguration}\PY{p}{(}\PY{p}{)}\PY{p}{\PYZob{}}
    \PY{n}{Configuration} \PY{n}{config}\PY{p}{;}
    \PY{n}{config}\PY{p}{.}\PY{n}{Parameter}\PY{o}{<}\PY{k+kt}{int}\PY{o}{>}\PY{p}{(}\PY{l+s}{"}\PY{l+s}{Seed}\PY{l+s}{"}\PY{p}{,}\PY{l+m+mi}{0}\PY{p}{)}\PY{p}{;}
    \PY{k}{return} \PY{n}{config}\PY{p}{;}
  \PY{p}{\PYZcb{}}

  \PY{c+c1}{// This method is what is used to actually configure the module.  }
  \PY{c+c1}{// The user has presumably set all the parameters and now we take}
  \PY{c+c1}{// them and do something useful with them.}
  \PY{k+kt}{void} \PY{n}{Initialize}\PY{p}{(}\PY{k}{const} \PY{n}{Configuration}\PY{o}{&} \PY{n}{config}\PY{p}{)}\PY{p}{\PYZob{}}
    \PY{n}{config}\PY{p}{.}\PY{n}{GetParameter}\PY{p}{(}\PY{l+s}{"}\PY{l+s}{Seed}\PY{l+s}{"}\PY{p}{,}\PY{n}{seed\PYZus{}}\PY{p}{)}\PY{p}{;}
    \PY{n}{log\PYZus{}info}\PY{p}{(}\PY{l+s}{"}\PY{l+s}{seeding random number generator with seed }\PY{l+s}{"} \PY{o}{<}\PY{o}{<} \PY{n}{seed\PYZus{}}\PY{p}{)}\PY{p}{;}
    \PY{n}{srand}\PY{p}{(}\PY{n}{seed\PYZus{}}\PY{p}{)}\PY{p}{;}
  \PY{p}{\PYZcb{}}

  \PY{c+c1}{// and here is implementing the methods promised by the interface}
  \PY{k+kt}{double} \PY{n}{Uniform}\PY{p}{(}\PY{k+kt}{double} \PY{n}{low}\PY{p}{,}\PY{k+kt}{double} \PY{n}{high}\PY{p}{)}\PY{p}{\PYZob{}}
    \PY{k+kt}{double} \PY{n}{rnd} \PY{o}{=} \PY{p}{(}\PY{k+kt}{double}\PY{p}{)}\PY{n}{rand}\PY{p}{(}\PY{p}{)} \PY{o}{/}\PY{p}{(}\PY{k+kt}{double}\PY{p}{)}\PY{n}{RAND\PYZus{}MAX}\PY{p}{;}
    \PY{k+kt}{double} \PY{n}{toReturn} \PY{o}{=} \PY{n}{rnd} \PY{o}{*}\PY{p}{(}\PY{n}{high} \PY{o}{-} \PY{n}{low}\PY{p}{)} \PY{o}{+} \PY{n}{low}\PY{p}{;}
    \PY{k}{return} \PY{n}{toReturn}\PY{p}{;}
  \PY{p}{\PYZcb{}}

 \PY{n+nl}{private:}
  \PY{k+kt}{int} \PY{n}{seed\PYZus{}}\PY{p}{;}
\PY{p}{\PYZcb{}}\PY{p}{;}

\PY{c+c1}{// This is a data structure which I fancy holds the calibration constants}
\PY{n}{class} \PY{n}{Calibration}
\PY{p}{\PYZob{}}
 \PY{n+nl}{public:}
  \PY{n}{std}\PY{o}{:}\PY{o}{:}\PY{n}{vector}\PY{o}{<}\PY{k+kt}{float}\PY{o}{>} \PY{n}{fakeCalibrationConstants\PYZus{}}\PY{p}{;}
\PY{p}{\PYZcb{}}\PY{p}{;}

\PY{c+c1}{// Here is a service which retrieves the calibration keyed off the}
\PY{c+c1}{// event run number}
\PY{n}{class} \PY{n}{CalibrationService}
\PY{p}{\PYZob{}}
 \PY{n+nl}{public:}
  \PY{k}{virtual} \PY{n}{Calibration}\PY{o}{&} \PY{n}{GetCalibration}\PY{p}{(}\PY{k+kt}{unsigned} \PY{n}{run}\PY{p}{)} \PY{o}{=} \PY{l+m+mi}{0}\PY{p}{;}

  \PY{k}{virtual} \PY{o}{~}\PY{n}{CalibrationService}\PY{p}{(}\PY{p}{)}\PY{p}{\PYZob{}}\PY{p}{\PYZcb{}}
\PY{p}{\PYZcb{}}\PY{p}{;}

\PY{c+c1}{// This is a dummy implementation, except I fancy that it goes off}
\PY{c+c1}{// and fetches the calibration constants from a DB}
\PY{n}{class} \PY{n}{DBCalibrationService} \PY{o}{:} \PY{n}{public} \PY{n}{CalibrationService}
\PY{p}{\PYZob{}}
 \PY{n+nl}{public:}
  \PY{k}{typedef} \PY{n}{CalibrationService} \PY{n}{Interface}\PY{p}{;}
  \PY{n}{Configuration} \PY{n}{DefaultConfiguration}\PY{p}{(}\PY{p}{)}\PY{p}{\PYZob{}}
    \PY{n}{Configuration} \PY{n}{config}\PY{p}{;}
    \PY{n}{config}\PY{p}{.}\PY{n}{Parameter}\PY{o}{<}\PY{n}{std}\PY{o}{:}\PY{o}{:}\PY{n}{string}\PY{o}{>}\PY{p}{(}\PY{l+s}{"}\PY{l+s}{server}\PY{l+s}{"}\PY{p}{,}\PY{l+s}{"}\PY{l+s}{mildb.umd.edu}\PY{l+s}{"}\PY{p}{)}\PY{p}{;}
    \PY{n}{config}\PY{p}{.}\PY{n}{Parameter}\PY{o}{<}\PY{n}{std}\PY{o}{:}\PY{o}{:}\PY{n}{string}\PY{o}{>}\PY{p}{(}\PY{l+s}{"}\PY{l+s}{uname}\PY{l+s}{"}\PY{p}{,}\PY{l+s}{"}\PY{l+s}{milagro}\PY{l+s}{"}\PY{p}{)}\PY{p}{;}
    \PY{n}{config}\PY{p}{.}\PY{n}{Parameter}\PY{o}{<}\PY{n}{std}\PY{o}{:}\PY{o}{:}\PY{n}{string}\PY{o}{>}\PY{p}{(}\PY{l+s}{"}\PY{l+s}{password}\PY{l+s}{"}\PY{p}{,}\PY{l+s}{"}\PY{l+s}{milagro}\PY{l+s}{"}\PY{p}{)}\PY{p}{;}
    \PY{k}{return} \PY{n}{config}\PY{p}{;}
  \PY{p}{\PYZcb{}}
  \PY{k+kt}{void} \PY{n}{Initialize}\PY{p}{(}\PY{k}{const} \PY{n}{Configuration}\PY{o}{&} \PY{n}{config}\PY{p}{)}\PY{p}{\PYZob{}}
    \PY{n}{config}\PY{p}{.}\PY{n}{GetParameter}\PY{p}{(}\PY{l+s}{"}\PY{l+s}{server}\PY{l+s}{"}\PY{p}{,}\PY{n}{server\PYZus{}}\PY{p}{)}\PY{p}{;}
    \PY{n}{config}\PY{p}{.}\PY{n}{GetParameter}\PY{p}{(}\PY{l+s}{"}\PY{l+s}{uname}\PY{l+s}{"}\PY{p}{,}\PY{n}{uname\PYZus{}}\PY{p}{)}\PY{p}{;}
    \PY{n}{config}\PY{p}{.}\PY{n}{GetParameter}\PY{p}{(}\PY{l+s}{"}\PY{l+s}{password}\PY{l+s}{"}\PY{p}{,}\PY{n}{password\PYZus{}}\PY{p}{)}\PY{p}{;}
  \PY{p}{\PYZcb{}}
  \PY{k+kt}{void} \PY{n}{Finish}\PY{p}{(}\PY{p}{)}\PY{p}{\PYZob{}}
    \PY{n}{log\PYZus{}info}\PY{p}{(}\PY{l+s}{"}\PY{l+s}{closing DB connection}\PY{l+s}{"}\PY{p}{)}\PY{p}{;}
  \PY{p}{\PYZcb{}}

  \PY{n}{Calibration}\PY{o}{&} \PY{n}{GetCalibration}\PY{p}{(}\PY{k+kt}{unsigned} \PY{n}{run}\PY{p}{)}\PY{p}{\PYZob{}}
    \PY{n}{log\PYZus{}info}\PY{p}{(}\PY{l+s}{"}\PY{l+s}{Pretend I'm reading calibration from a DB here}\PY{l+s}{"}\PY{p}{)}\PY{p}{;}
    \PY{n}{log\PYZus{}info}\PY{p}{(}\PY{l+s}{"}\PY{l+s}{Connecting to server:}\PY{l+s}{"} \PY{o}{<}\PY{o}{<} \PY{n}{server\PYZus{}}
             \PY{o}{<}\PY{o}{<} \PY{l+s}{"}\PY{l+s}{ with uname:}\PY{l+s}{"} \PY{o}{<}\PY{o}{<} \PY{n}{uname\PYZus{}}
             \PY{o}{<}\PY{o}{<} \PY{l+s}{"}\PY{l+s}{ and password:}\PY{l+s}{"} \PY{o}{<}\PY{o}{<} \PY{n}{password\PYZus{}}\PY{p}{)}\PY{p}{;}
    \PY{k}{return} \PY{n}{calibration\PYZus{}}\PY{p}{;} \PY{c+c1}{// pretend we fetch it from a DB here}
  \PY{p}{\PYZcb{}}
 \PY{n+nl}{private:}
  \PY{n}{Calibration} \PY{n}{calibration\PYZus{}}\PY{p}{;}
  \PY{n}{std}\PY{o}{:}\PY{o}{:}\PY{n}{string} \PY{n}{server\PYZus{}}\PY{p}{;}
  \PY{n}{std}\PY{o}{:}\PY{o}{:}\PY{n}{string} \PY{n}{uname\PYZus{}}\PY{p}{;}
  \PY{n}{std}\PY{o}{:}\PY{o}{:}\PY{n}{string} \PY{n}{password\PYZus{}}\PY{p}{;}
\PY{p}{\PYZcb{}}\PY{p}{;}

\PY{c+c1}{// Here, I fancy that we are getting the calibration constants from }
\PY{c+c1}{// a flat file}
\PY{n}{class} \PY{n}{FlatFileCalibrationService} \PY{o}{:} \PY{n}{public} \PY{n}{CalibrationService}
\PY{p}{\PYZob{}}
 \PY{n+nl}{public:}
  \PY{k}{typedef} \PY{n}{CalibrationService} \PY{n}{Interface}\PY{p}{;}

  \PY{n}{Configuration} \PY{n}{DefaultConfiguration}\PY{p}{(}\PY{p}{)}\PY{p}{\PYZob{}}
    \PY{n}{Configuration} \PY{n}{config}\PY{p}{;}
    \PY{n}{config}\PY{p}{.}\PY{n}{Parameter}\PY{o}{<}\PY{n}{std}\PY{o}{:}\PY{o}{:}\PY{n}{string}\PY{o}{>}\PY{p}{(}\PY{l+s}{"}\PY{l+s}{filename}\PY{l+s}{"}\PY{p}{,}\PY{l+s}{"}\PY{l+s}{calibration.dat}\PY{l+s}{"}\PY{p}{)}\PY{p}{;}
    \PY{k}{return} \PY{n}{config}\PY{p}{;}
  \PY{p}{\PYZcb{}}

  \PY{n}{Calibration}\PY{o}{&} \PY{n}{GetCalibration}\PY{p}{(}\PY{k+kt}{unsigned} \PY{n}{run}\PY{p}{)}\PY{p}{\PYZob{}}
    \PY{n}{log\PYZus{}info}\PY{p}{(}\PY{l+s}{"}\PY{l+s}{pretend I'm reading calibration from a flat file here}\PY{l+s}{"}\PY{p}{)}\PY{p}{;}
    \PY{k}{return} \PY{n}{calibration\PYZus{}}\PY{p}{;} \PY{c+c1}{// pretend we fetch it from a DB here}
  \PY{p}{\PYZcb{}}
 \PY{n+nl}{private:}
  \PY{n}{Calibration} \PY{n}{calibration\PYZus{}}\PY{p}{;}
\PY{p}{\PYZcb{}}\PY{p}{;}


\PY{c+c1}{// This is an example 'Source' which is the first Module added to the }
\PY{c+c1}{// framework.  This source fills the data structure.  Typically the source}
\PY{c+c1}{// would get data from a file or a socket, but this one makes it up}
\PY{c+c1}{// from scratch.  The difference between a source and a Module is just}
\PY{c+c1}{// function 'Next'.  The framework processes until Next returns a null}
\PY{c+c1}{// pointer}
\PY{n}{class} \PY{n}{ExampleSource} \PY{o}{:} \PY{n}{public} \PY{n}{Source}
\PY{p}{\PYZob{}}
\PY{n+nl}{public:}
  \PY{k}{typedef} \PY{n}{Source} \PY{n}{Interface}\PY{p}{;}

  \PY{n}{ExampleSource}\PY{p}{(}\PY{p}{)}\PY{p}{\PYZob{}}\PY{n}{events\PYZus{}} \PY{o}{=} \PY{l+m+mi}{0}\PY{p}{;}\PY{p}{\PYZcb{}}

  \PY{n}{Configuration} \PY{n}{DefaultConfiguration}\PY{p}{(}\PY{p}{)}\PY{p}{\PYZob{}}
    \PY{n}{Configuration} \PY{n}{config}\PY{p}{;}

    \PY{c+c1}{// not supplying a default value means the framework will complain}
    \PY{c+c1}{// if the user doesn't set a value for 'input'}
    \PY{n}{config}\PY{p}{.}\PY{n}{Parameter}\PY{o}{<}\PY{n}{std}\PY{o}{:}\PY{o}{:}\PY{n}{string}\PY{o}{>}\PY{p}{(}\PY{l+s}{"}\PY{l+s}{input}\PY{l+s}{"}\PY{p}{)}\PY{p}{;}
    \PY{n}{config}\PY{p}{.}\PY{n}{Parameter}\PY{o}{<}\PY{k+kt}{int}\PY{o}{>}\PY{p}{(}\PY{l+s}{"}\PY{l+s}{maxevents}\PY{l+s}{"}\PY{p}{,}\PY{l+m+mi}{10}\PY{p}{)}\PY{p}{;}
    \PY{k}{return} \PY{n}{config}\PY{p}{;}
  \PY{p}{\PYZcb{}}

  \PY{c+c1}{// This function is called before 'Process' is called.  This eventually}
  \PY{c+c1}{// will be where the module gets configured}
  \PY{k+kt}{void} \PY{n}{Initialize}\PY{p}{(}\PY{k}{const} \PY{n}{Configuration}\PY{o}{&} \PY{n}{config}\PY{p}{)}
  \PY{p}{\PYZob{}}
    \PY{n}{config}\PY{p}{.}\PY{n}{GetParameter}\PY{p}{(}\PY{l+s}{"}\PY{l+s}{input}\PY{l+s}{"}\PY{p}{,}\PY{n}{infile\PYZus{}}\PY{p}{)}\PY{p}{;}
    \PY{n}{config}\PY{p}{.}\PY{n}{GetParameter}\PY{p}{(}\PY{l+s}{"}\PY{l+s}{maxevents}\PY{l+s}{"}\PY{p}{,}\PY{n}{maxEvents\PYZus{}}\PY{p}{)}\PY{p}{;}
    \PY{n}{log\PYZus{}info}\PY{p}{(}\PY{l+s}{"}\PY{l+s}{opening the imaginary file '}\PY{l+s}{"} \PY{o}{<}\PY{o}{<} \PY{n}{infile\PYZus{}} \PY{o}{<}\PY{o}{<} \PY{l+s}{"}\PY{l+s}{'}\PY{l+s}{"}\PY{p}{)}\PY{p}{;}
    \PY{n}{log\PYZus{}info}\PY{p}{(}\PY{l+s}{"}\PY{l+s}{processing }\PY{l+s}{"} \PY{o}{<}\PY{o}{<} \PY{n}{maxEvents\PYZus{}} \PY{o}{<}\PY{o}{<} \PY{l+s}{"}\PY{l+s}{ events}\PY{l+s}{"}\PY{p}{)}\PY{p}{;}
  \PY{p}{\PYZcb{}}

  \PY{c+c1}{// This method gets called for each event.  The EventPtr structure }
  \PY{c+c1}{// already points somewhere valid and we just fill it.}
  \PY{n}{BagPtr} \PY{n}{Next}\PY{p}{(}\PY{p}{)}\PY{p}{\PYZob{}}
    \PY{k}{if}\PY{p}{(}\PY{n}{events\PYZus{}} \PY{o}{>}\PY{o}{=} \PY{n}{maxEvents\PYZus{}}\PY{p}{)}
      \PY{k}{return} \PY{n}{BagPtr}\PY{p}{(}\PY{p}{)}\PY{p}{;}
    \PY{n}{BagPtr} \PY{n}{b}\PY{p}{(}\PY{n}{new} \PY{n}{Bag}\PY{p}{(}\PY{p}{)}\PY{p}{)}\PY{p}{;}
    \PY{n}{log\PYZus{}info}\PY{p}{(}\PY{l+s}{"}\PY{l+s}{========================================}\PY{l+s+se}{\PYZbs{}n}\PY{l+s}{"}
             \PY{o}{<}\PY{o}{<} \PY{l+s}{"}\PY{l+s}{'reading' event }\PY{l+s}{"} \PY{o}{<}\PY{o}{<} \PY{n}{events\PYZus{}} 
             \PY{o}{<}\PY{o}{<} \PY{l+s}{"}\PY{l+s}{ from file }\PY{l+s}{"} \PY{o}{<}\PY{o}{<} \PY{n}{infile\PYZus{}}\PY{p}{)}\PY{p}{;} 
    \PY{n}{EventHeaderPtr} \PY{n}{header}\PY{p}{(}\PY{n}{new} \PY{n}{EventHeader}\PY{p}{(}\PY{p}{)}\PY{p}{)}\PY{p}{;}
    \PY{n}{header}\PY{o}{-}\PY{o}{>}\PY{n}{event\PYZus{}number} \PY{o}{=} \PY{n}{events\PYZus{}}\PY{p}{;}
    \PY{n}{header}\PY{o}{-}\PY{o}{>}\PY{n}{run\PYZus{}number} \PY{o}{=} \PY{l+m+mi}{12345}\PY{p}{;} 

    \PY{n}{b}\PY{o}{-}\PY{o}{>}\PY{n}{Put}\PY{p}{(}\PY{l+s}{"}\PY{l+s}{Header}\PY{l+s}{"}\PY{p}{,}\PY{n}{header}\PY{p}{)}\PY{p}{;}

    \PY{n}{HitDataPtr} \PY{n}{hitData}\PY{p}{(}\PY{n}{new} \PY{n}{HitData}\PY{p}{(}\PY{p}{)}\PY{p}{)}\PY{p}{;}
    \PY{k+kt}{int} \PY{n}{hitChannels} \PY{o}{=} \PY{n}{rand}\PY{p}{(}\PY{p}{)} \PY{o}{%} \PY{l+m+mi}{50}\PY{p}{;}
    \PY{n}{hitData}\PY{o}{-}\PY{o}{>}\PY{n}{hit\PYZus{}channels}\PY{p}{.}\PY{n}{resize}\PY{p}{(}\PY{n}{hitChannels}\PY{p}{)}\PY{p}{;}
    \PY{n}{hitData}\PY{o}{-}\PY{o}{>}\PY{n}{hit\PYZus{}times}\PY{p}{.}\PY{n}{resize}\PY{p}{(}\PY{n}{hitChannels}\PY{p}{)}\PY{p}{;}
    \PY{k}{for}\PY{p}{(}\PY{k+kt}{int} \PY{n}{i} \PY{o}{=} \PY{l+m+mi}{0} \PY{p}{;} \PY{n}{i} \PY{o}{<} \PY{n}{hitChannels} \PY{p}{;} \PY{n}{i}\PY{o}{+}\PY{o}{+}\PY{p}{)}\PY{p}{\PYZob{}}
      \PY{n}{hitData}\PY{o}{-}\PY{o}{>}\PY{n}{hit\PYZus{}channels}\PY{p}{[}\PY{n}{i}\PY{p}{]} \PY{o}{=} \PY{n}{rand}\PY{p}{(}\PY{p}{)} \PY{o}{%} \PY{l+m+mi}{900}\PY{p}{;}
      \PY{n}{hitData}\PY{o}{-}\PY{o}{>}\PY{n}{hit\PYZus{}times}\PY{p}{[}\PY{n}{i}\PY{p}{]} \PY{o}{=} \PY{p}{(}\PY{k+kt}{float}\PY{p}{)}\PY{n}{rand}\PY{p}{(}\PY{p}{)}\PY{o}{/}\PY{p}{(}\PY{k+kt}{float}\PY{p}{)}\PY{n}{RAND\PYZus{}MAX}\PY{p}{;}
    \PY{p}{\PYZcb{}}
    
    \PY{n}{b}\PY{o}{-}\PY{o}{>}\PY{n}{Put}\PY{p}{(}\PY{l+s}{"}\PY{l+s}{RawHitData}\PY{l+s}{"}\PY{p}{,}\PY{n}{hitData}\PY{p}{)}\PY{p}{;}

    \PY{n}{events\PYZus{}} \PY{o}{+}\PY{o}{=} \PY{l+m+mi}{1}\PY{p}{;} 

    \PY{k}{return} \PY{n}{b}\PY{p}{;}
  \PY{p}{\PYZcb{}}

  \PY{c+c1}{// This method is called when all is done}
  \PY{k+kt}{void} \PY{n}{Finish}\PY{p}{(}\PY{p}{)}\PY{p}{\PYZob{}}
    \PY{n}{log\PYZus{}info}\PY{p}{(}\PY{l+s}{"}\PY{l+s}{closing the imaginary file}\PY{l+s}{"}\PY{p}{)}\PY{p}{;}
  \PY{p}{\PYZcb{}}
\PY{n+nl}{private:}
  \PY{k+kt}{int} \PY{n}{events\PYZus{}}\PY{p}{;}
  \PY{k+kt}{int} \PY{n}{maxEvents\PYZus{}}\PY{p}{;}
  \PY{n}{std}\PY{o}{:}\PY{o}{:}\PY{n}{string} \PY{n}{infile\PYZus{}}\PY{p}{;}
\PY{p}{\PYZcb{}}\PY{p}{;}

\PY{c+c1}{// Another dummy example.  This one does a fake calibration}
\PY{n}{class} \PY{n}{CalibrationModule} \PY{o}{:} \PY{n}{public} \PY{n}{Module}
\PY{p}{\PYZob{}}
 \PY{n+nl}{public:}
  \PY{k}{typedef} \PY{n}{Module} \PY{n}{Interface}\PY{p}{;}

  \PY{n}{Result} \PY{n}{Process}\PY{p}{(}\PY{n}{BagPtr} \PY{n}{b}\PY{p}{)}\PY{p}{\PYZob{}}
    \PY{c+c1}{// This shows how to fetch the calibration serice and use it.  }
    \PY{c+c1}{// N.B. The complete ignorance of whether this comes from a DB}
    \PY{c+c1}{// or a file.}
    \PY{k}{const} \PY{n}{EventHeader}\PY{o}{&} \PY{n}{head} \PY{o}{=} \PY{n}{b}\PY{o}{-}\PY{o}{>}\PY{n}{Get}\PY{o}{<}\PY{n}{EventHeader}\PY{o}{>}\PY{p}{(}\PY{l+s}{"}\PY{l+s}{Header}\PY{l+s}{"}\PY{p}{)}\PY{p}{;}
    \PY{n}{Calibration}\PY{o}{&} \PY{n}{calib} \PY{o}{=}  
      \PY{n}{GetService}\PY{o}{<}\PY{n}{CalibrationService}\PY{o}{>}\PY{p}{(}\PY{l+s}{"}\PY{l+s}{calib}\PY{l+s}{"}\PY{p}{)}\PY{p}{.}\PY{n}{GetCalibration}\PY{p}{(}\PY{n}{head}\PY{p}{.}\PY{n}{run\PYZus{}number}\PY{p}{)}\PY{p}{;}
    
    \PY{n}{log\PYZus{}info}\PY{p}{(}\PY{l+s}{"}\PY{l+s}{calibrating the event with }\PY{l+s}{"}
             \PY{o}{<}\PY{o}{<} \PY{n}{calib}\PY{p}{.}\PY{n}{fakeCalibrationConstants\PYZus{}}\PY{p}{.}\PY{n}{size}\PY{p}{(}\PY{p}{)} \PY{o}{<}\PY{o}{<} \PY{l+s}{"}\PY{l+s}{ constants}\PY{l+s}{"} \PY{p}{)}
    \PY{k}{return} \PY{n}{Continue}\PY{p}{;}
  \PY{p}{\PYZcb{}}
\PY{p}{\PYZcb{}}\PY{p}{;}

\PY{c+c1}{// A fake Reconstruction module... COM technique}
\PY{n}{class} \PY{n}{ReconstructionModule\PYZus{}COM} \PY{o}{:} \PY{n}{public} \PY{n}{Module}
\PY{p}{\PYZob{}}
 \PY{n+nl}{public:}
  \PY{k}{typedef} \PY{n}{Module} \PY{n}{Interface}\PY{p}{;}

  \PY{n}{Result} \PY{n}{Process}\PY{p}{(}\PY{n}{BagPtr} \PY{n}{e}\PY{p}{)}\PY{p}{\PYZob{}}
    \PY{n}{RandomNumberService}\PY{o}{&} \PY{n}{random} \PY{o}{=} \PY{n}{GetService}\PY{o}{<}\PY{n}{RandomNumberService}\PY{o}{>}\PY{p}{(}\PY{l+s}{"}\PY{l+s}{rand}\PY{l+s}{"}\PY{p}{)}\PY{p}{;} 
    \PY{n}{log\PYZus{}info}\PY{p}{(}\PY{l+s}{"}\PY{l+s}{reconstructing the event with COM technique}\PY{l+s}{"}\PY{p}{)}\PY{p}{;} 

    \PY{n}{CoreReconstructionPtr} \PY{n}{core}\PY{p}{(}\PY{n}{new} \PY{n}{CoreReconstruction}\PY{p}{(}\PY{p}{)}\PY{p}{)}\PY{p}{;}
    \PY{n}{core}\PY{o}{-}\PY{o}{>}\PY{n}{corex} \PY{o}{=}\PY{n}{random}\PY{p}{.}\PY{n}{Uniform}\PY{p}{(}\PY{o}{-}\PY{l+m+mi}{2000}\PY{p}{,}\PY{l+m+mi}{2000}\PY{p}{)}\PY{p}{;} 
    \PY{n}{core}\PY{o}{-}\PY{o}{>}\PY{n}{corey} \PY{o}{=}\PY{n}{random}\PY{p}{.}\PY{n}{Uniform}\PY{p}{(}\PY{o}{-}\PY{l+m+mi}{2000}\PY{p}{,}\PY{l+m+mi}{2000}\PY{p}{)}\PY{p}{;} 

    \PY{n}{e}\PY{o}{-}\PY{o}{>}\PY{n}{Put}\PY{p}{(}\PY{l+s}{"}\PY{l+s}{Core\PYZus{}COM}\PY{l+s}{"}\PY{p}{,}\PY{n}{core}\PY{p}{)}\PY{p}{;}
    
    \PY{k}{return} \PY{n}{Continue}\PY{p}{;}
  \PY{p}{\PYZcb{}}
\PY{p}{\PYZcb{}}\PY{p}{;}

\PY{c+c1}{// A fake Reconstruction module... Gauss technique}
\PY{n}{class} \PY{n}{ReconstructionModule\PYZus{}Gauss} \PY{o}{:} \PY{n}{public} \PY{n}{Module}
\PY{p}{\PYZob{}}
 \PY{n+nl}{public:}
  \PY{k}{typedef} \PY{n}{Module} \PY{n}{Interface}\PY{p}{;}

  \PY{n}{Result} \PY{n}{Process}\PY{p}{(}\PY{n}{BagPtr} \PY{n}{e}\PY{p}{)}\PY{p}{\PYZob{}}
    \PY{n}{RandomNumberService}\PY{o}{&} \PY{n}{random} \PY{o}{=} \PY{n}{GetService}\PY{o}{<}\PY{n}{RandomNumberService}\PY{o}{>}\PY{p}{(}\PY{l+s}{"}\PY{l+s}{rand}\PY{l+s}{"}\PY{p}{)}\PY{p}{;} 
    \PY{n}{log\PYZus{}info}\PY{p}{(}\PY{l+s}{"}\PY{l+s}{reconstructing the event with Gauss technique}\PY{l+s}{"}\PY{p}{)}\PY{p}{;} 

    \PY{n}{CoreReconstructionPtr} \PY{n}{core}\PY{p}{(}\PY{n}{new} \PY{n}{CoreReconstruction}\PY{p}{(}\PY{p}{)}\PY{p}{)}\PY{p}{;}
    \PY{n}{core}\PY{o}{-}\PY{o}{>}\PY{n}{corex} \PY{o}{=}\PY{n}{random}\PY{p}{.}\PY{n}{Uniform}\PY{p}{(}\PY{o}{-}\PY{l+m+mi}{2000}\PY{p}{,}\PY{l+m+mi}{2000}\PY{p}{)}\PY{p}{;} 
    \PY{n}{core}\PY{o}{-}\PY{o}{>}\PY{n}{corey} \PY{o}{=}\PY{n}{random}\PY{p}{.}\PY{n}{Uniform}\PY{p}{(}\PY{o}{-}\PY{l+m+mi}{2000}\PY{p}{,}\PY{l+m+mi}{2000}\PY{p}{)}\PY{p}{;} 

    \PY{n}{e}\PY{o}{-}\PY{o}{>}\PY{n}{Put}\PY{p}{(}\PY{l+s}{"}\PY{l+s}{Core\PYZus{}Gauss}\PY{l+s}{"}\PY{p}{,}\PY{n}{core}\PY{p}{)}\PY{p}{;}

    \PY{k}{return} \PY{n}{Continue}\PY{p}{;}
  \PY{p}{\PYZcb{}}
\PY{p}{\PYZcb{}}\PY{p}{;}

\PY{c+c1}{// A module which prints the values in the event structure}
\PY{n}{class} \PY{n}{PrintingModule} \PY{o}{:} \PY{n}{public} \PY{n}{Module}
\PY{p}{\PYZob{}}
 \PY{n+nl}{public:}
  \PY{k}{typedef} \PY{n}{Module} \PY{n}{Interface}\PY{p}{;}

  \PY{n}{Result} \PY{n}{Process}\PY{p}{(}\PY{n}{BagPtr} \PY{n}{b}\PY{p}{)}\PY{p}{\PYZob{}}
    \PY{n}{log\PYZus{}info}\PY{p}{(}\PY{l+s}{"}\PY{l+s}{--------------------------------}\PY{l+s+se}{\PYZbs{}n}\PY{l+s}{"} \PY{o}{<}\PY{o}{<} \PY{o}{*}\PY{n}{b}
             \PY{o}{<}\PY{o}{<} \PY{l+s}{"}\PY{l+s}{--------------------------------}\PY{l+s+se}{\PYZbs{}n}\PY{l+s}{"}\PY{p}{)}\PY{p}{;}

    \PY{k}{return} \PY{n}{Continue}\PY{p}{;}
  \PY{p}{\PYZcb{}}
\PY{p}{\PYZcb{}}\PY{p}{;}

\PY{c+cp}{#}\PY{c+cp}{endif}
\end{Verbatim}

\begin{Verbatim}[commandchars=\\\{\}]
\PY{c}{# importing the needed classes and functions}
\PY{k+kn}{from} \PY{n+nn}{HAWCNest} \PY{k+kn}{import} \PY{n}{HAWCNest}
\PY{k+kn}{from} \PY{n+nn}{HAWCNest} \PY{k+kn}{import} \PY{n}{load}
\PY{k+kn}{from} \PY{n+nn}{HAWCNest} \PY{k+kn}{import} \PY{n}{GetService\PYZus{}MainLoop}

\PY{c}{# loading shared libraries that might be needed for processing}
\PY{n}{load}\PY{p}{(}\PY{l+s}{"}\PY{l+s}{libhawcnest}\PY{l+s}{"}\PY{p}{)}
\PY{n}{load}\PY{p}{(}\PY{l+s}{"}\PY{l+s}{libexample-hawcnest}\PY{l+s}{"}\PY{p}{)}

\PY{c}{# This is nearly identical to the C++ interface, except it uses python}
\PY{c}{# syntax}
\PY{n}{nest} \PY{o}{=} \PY{n}{HAWCNest}\PY{p}{(}\PY{p}{)}

\PY{c}{# adding and configuring services}
\PY{n}{nest}\PY{o}{.}\PY{n}{Service}\PY{p}{(}\PY{l+s}{"}\PY{l+s}{STDRandomNumberService}\PY{l+s}{"}\PY{p}{,}\PY{l+s}{"}\PY{l+s}{rand}\PY{l+s}{"}\PY{p}{)}\PY{p}{(}
    \PY{p}{(}\PY{l+s}{"}\PY{l+s}{Seed}\PY{l+s}{"}\PY{p}{,}\PY{l+m+mi}{12345}\PY{p}{)}
    \PY{p}{)}
\PY{n}{nest}\PY{o}{.}\PY{n}{Service}\PY{p}{(}\PY{l+s}{"}\PY{l+s}{DBCalibrationService}\PY{l+s}{"}\PY{p}{,}\PY{l+s}{"}\PY{l+s}{calib}\PY{l+s}{"}\PY{p}{)}\PY{p}{(}
    \PY{p}{(}\PY{l+s}{"}\PY{l+s}{server}\PY{l+s}{"}\PY{p}{,}\PY{l+s}{"}\PY{l+s}{mildb.umd.edu}\PY{l+s}{"}\PY{p}{)}
    \PY{p}{(}\PY{l+s}{"}\PY{l+s}{uname}\PY{l+s}{"}\PY{p}{,}\PY{l+s}{"}\PY{l+s}{milagro}\PY{l+s}{"}\PY{p}{)}
    \PY{p}{(}\PY{l+s}{"}\PY{l+s}{password}\PY{l+s}{"}\PY{p}{,}\PY{l+s}{"}\PY{l+s}{topsecret}\PY{l+s}{"}\PY{p}{)}
    \PY{p}{)}
\PY{n}{nest}\PY{o}{.}\PY{n}{Service}\PY{p}{(}\PY{l+s}{"}\PY{l+s}{ExampleSource}\PY{l+s}{"}\PY{p}{,}\PY{l+s}{"}\PY{l+s}{source}\PY{l+s}{"}\PY{p}{)}\PY{p}{(}
    \PY{p}{(}\PY{l+s}{"}\PY{l+s}{input}\PY{l+s}{"}\PY{p}{,}\PY{l+s}{"}\PY{l+s}{myinputfile.dat}\PY{l+s}{"}\PY{p}{)}
    \PY{p}{(}\PY{l+s}{"}\PY{l+s}{maxevents}\PY{l+s}{"}\PY{p}{,}\PY{l+m+mi}{20}\PY{p}{)}
    \PY{p}{)}
\PY{n}{nest}\PY{o}{.}\PY{n}{Service}\PY{p}{(}\PY{l+s}{"}\PY{l+s}{CalibrationModule}\PY{l+s}{"}\PY{p}{,}\PY{l+s}{"}\PY{l+s}{calibmodule}\PY{l+s}{"}\PY{p}{)}
\PY{n}{nest}\PY{o}{.}\PY{n}{Service}\PY{p}{(}\PY{l+s}{"}\PY{l+s}{ReconstructionModule\PYZus{}COM}\PY{l+s}{"}\PY{p}{,}\PY{l+s}{"}\PY{l+s}{reco\PYZus{}com}\PY{l+s}{"}\PY{p}{)}
\PY{n}{nest}\PY{o}{.}\PY{n}{Service}\PY{p}{(}\PY{l+s}{"}\PY{l+s}{ReconstructionModule\PYZus{}Gauss}\PY{l+s}{"}\PY{p}{,}\PY{l+s}{"}\PY{l+s}{reco\PYZus{}gauss}\PY{l+s}{"}\PY{p}{)}
\PY{n}{nest}\PY{o}{.}\PY{n}{Service}\PY{p}{(}\PY{l+s}{"}\PY{l+s}{PrintingModule}\PY{l+s}{"}\PY{p}{,}\PY{l+s}{"}\PY{l+s}{print}\PY{l+s}{"}\PY{p}{)}
\PY{n}{nest}\PY{o}{.}\PY{n}{Service}\PY{p}{(}\PY{l+s}{"}\PY{l+s}{SequentialMainLoop}\PY{l+s}{"}\PY{p}{,}\PY{l+s}{"}\PY{l+s}{mainloop}\PY{l+s}{"}\PY{p}{)}\PY{p}{(}
    \PY{p}{(}\PY{l+s}{"}\PY{l+s}{source}\PY{l+s}{"}\PY{p}{,}\PY{l+s}{"}\PY{l+s}{source}\PY{l+s}{"}\PY{p}{)}
    \PY{p}{(}\PY{l+s}{"}\PY{l+s}{modulechain}\PY{l+s}{"}\PY{p}{,}\PY{p}{[}\PY{l+s}{"}\PY{l+s}{print}\PY{l+s}{"}\PY{p}{,}\PY{l+s}{"}\PY{l+s}{calibmodule}\PY{l+s}{"}\PY{p}{,}\PY{l+s}{"}\PY{l+s}{reco\PYZus{}com}\PY{l+s}{"}\PY{p}{,}
                    \PY{l+s}{"}\PY{l+s}{reco\PYZus{}gauss}\PY{l+s}{"}\PY{p}{,}\PY{l+s}{"}\PY{l+s}{print}\PY{l+s}{"}\PY{p}{,}\PY{l+s}{"}\PY{l+s}{mymodule}\PY{l+s}{"}\PY{p}{]}\PY{p}{)}\PY{p}{,}
    \PY{p}{)}
\PY{n}{nest}\PY{o}{.}\PY{n}{Configure}\PY{p}{(}\PY{p}{)}
\PY{n}{main} \PY{o}{=} \PY{n}{GetService\PYZus{}MainLoop}\PY{p}{(}\PY{l+s}{"}\PY{l+s}{mainloop}\PY{l+s}{"}\PY{p}{)}
\PY{n}{loop}\PY{o}{.}\PY{n}{Execute}\PY{p}{(}\PY{p}{)}
\PY{n}{nest}\PY{o}{.}\PY{n}{Finish}\PY{p}{(}\PY{p}{)}
\end{Verbatim}


\newpage
\section{Design Goals and Tour of the HAWCNest Code}

This section is intended to walk through the main design objectives, answer
questions about why certain things are done, and give an outline to the
implementation to start somebody off hacking the framework itself.  Not 
every wrinkle of the code, just the major features. This section also
mentions places where specific improvements are planned.

\subsection{Major Design Goals}

This section describes some of the major design goals. It goes without
saying that the framework is supposed to handle the addition and 
retrieval of Services, that it behave largely like described in this 
document. This section is mostly to enumerate some very specific goals
which drive some of the most frequently asked design questions. 
In this question, we're after an answer to the question: ``Why not do it
{\emph this} way.''  

The first major design goal was to de-couple the execution of event processing
modules from the core framework. Event processing is the most common
anticipated usage of HAWCNest, and there would be no particular harm in
incorporating the event processing model into the core framework. Nevertheless,
it was a major aim to include the event processing as a special case 
of the framework usage.  

A second major goal was that Services would not have to be invasively 
modified to add them to the framework. That is, implementations 
would not be required to inherit from
a common base class. We don't want to have a global Service class that
has virtual implementations of {\tt DefaultConfiguration}, {\tt Initialize}, 
and {\tt Finish}.  
We aim that a ROOT {\tt TRandom} object could be a service in the
framework if we needed it to; we shouldn't have to wrap every external
class that we find useful.  This design goal is not yet met, but the
foundation is laid.

A third major goal is that the framework could be configured from xml and
python interfaces. This goal requires that the code be able to generate
a service given only a string that is the service's type.

We will refer back to these design goals as we tour the code.

Finally, a note on the code layout. The include files are organized into
three directories. {\tt include/hawcnest/} includes all of the files
that are in the basic HAWCNest API. These are the classes that make up
the public interface to HAWCNest. Code in the {\tt impl}
directory is not part of the formal API. For compilation, many of these
declarations have to be available to the compiler, but they are not part
of the API. Finally, code in the {\tt processing} subdirectory is the
code related to the event processing model. It's distribued with 
HAWCNest, but it is included in a separate directory to emphasize that it
is logically distinct from the core framework.

\subsection{Core Framework}

The place to start the tour is the {\tt HAWCNest.h} file. 

The place to start is the definition of the {\tt Service<ServiceType>(...)}
method. This is where new services are added to the framework and 
illuminates what's going on.

When a new service of type {\tt ServiceType} is added, we do two important
things. First, we store a pointer to that service's {\tt Interface} in the 
global
{\tt ServiceLifetimeControl} for that {\tt Interface}. 
That class is a glorified 
{\tt map<string,InterfacePtr>} and is used by the {\tt GetSerivce}
function when a particular service is requested.  
It's a small class in {\tt impl/ServiceImpl.h}. The
actual {\tt map<string,InterfacePtr>} is stored as a static object within
a static method. It's a little trick to get around the need to specifically
enumerate static data to get it linked.
The retrieval of services is implemented in {\tt GetService<Interface>}
in {\tt Service.h}. 
Note that since {\tt ServiceLifetimeControl} is templated by the interface
type, there is no requirement that services inherit from a common base class.
Also, the retrieval of services requires the {\tt Interface} type and 
will fail if the full class name is used instead of the Interface type.

Second, when adding a new service, we store a 
{\tt ServiceWrapperBasePtr}
within the {\tt HAWCNest} class. 
{\tt ServiceWrapperBase} is a class which has virtual
 life-cycle functions that the HAWCNest services support, {\tt Initialize},
{\tt DefaultConfiguration}, and {\tt Finish}. A subclass, 
{\tt ServiceWrapper<ServiceType>}, just forwards the calls to the 
actual service instance. In this way, we can just have a long list
of {\tt ServiceWrapperBasePtr}s and call the various life-cycle functions
on them. This is the {\tt services\_} data member of {\tt HAWCNest}.
There is one perhaps confusing part of {\tt ServiceWrapper}. If you look at 
the implementation, you will notice that the method XXX is called via
a CallXXX free-standing function.  These functions are implemented in 
{\tt impl/Has.h}. They are implemented with a bit of 
compile-time magic that calls one function if WrappedType has the XXX method
and one if it doesn't.  This allows us to not require Service types to 
implement all of these required methods. In the future, it will be 
possible to re-implement the CallXXX methods so that any type can be used
as a service without the invasive requirement that it implement certain 
methods.

That's the main framework. Services are added via HAWCNest's 
{\tt Service}, stored
in the {\tt ServiceLifetimeControl} and then configured and available for use.
The remaining sections in the tour are ancilliary features.

\subsection{Logging Interface}

The logging interface is in the file {\tt Logging.h}.  The imlementation is 
pretty straightforward. Logging statements are formatted with some compile-time
information on where the logging statement comes from. If {\tt NDEBUG}
is defined, the trace, debug and info levels are compiled out completely.
At the moment, the logging statements are just directed to stderr
and that will have to be back-ended with something more useful later.

\subsection{Run-Time Module Types and the Python Interface}

The astute reader will notice that {\tt HAWCNest::Service} 
requires a service type
as the template argument. The problem with this is that it requires a full
C++ compiler to parse. In some cases, most notably the case of configuring
HAWCNest from python code, the type will be a string and not a type. 
In order to make something like this work, we need to outfit the framework
with some way to convert a string into a type. This is done via
a macro {\tt REGISTER\_SERVICE} included in the {\tt RegisterService.h} header.
Following your nose, the implementation is in {\tt impl/ServiceRegistry.h}.
The macro creates a global object in an anonymous namespace and the
constructor of that object is what registers the service. You can follow 
your nose and the comments from there. When one of these objects is created,
it registers the service type and 'teaches' the global {\tt ServiceRegistry}
how to add one of those classes to {\tt HAWCNest}.  The implementation
of the non-templated {\tt Service} method of {\tt HAWCNest} uses this
global service registry to do the service addition.

Given the ability to add services by string rather than by type, the 
implementation of the python interface is done rather straightforwardly 
using boost::python in {\tt HAWCNest\_python.cc}.  It is important to note
that the final HAWCNest class that is exposed to python is wrapped a bit 
further by the pure python code in {\tt HAWCNest.py}. Those two, the pure
C++ and the pure python, cooperate to give the python interface.

\subsection{Event Processing}

The event processing is not part of the core framework, but is distributed
with HAWCNest since it is the main usage forseen.  The code is included
in the {\tt processing/} subdirectory.  Since the event processing model
is dealt with fairly completely in Section \ref{processingsection}, it 
is not repeated here except to say that additional useful main loop 
implementations could be added when we understand better what topologies
are needed.


\end{document}
